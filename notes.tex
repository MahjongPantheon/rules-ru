\section{Приложения}

\subsection{Краткая сводка отличий от популярных наборов правил}

Данные правила основаны на правилах Европейской ассоциации маджонга, со следующими отличиями:

\begin{itemize}
	\item Акадоры: есть;
	\item Абортивные ничьи: есть, кроме абортивной ничьей на тройном роне;
	\item Счетный якуман: есть;
	\item Сложение якуманов: есть.
\end{itemize}

Далее дана чуть более подробная таблица отличий текущего регламента (RRC) от правил EMA и WRC.

\noindent\begin{tabularx}{\linewidth}{L{0.4\linewidth}L{0.15\linewidth}L{0.15\linewidth}L{0.15\linewidth}}
	\caption{Отличия правил} \\
	\toprule
	\textbf{Правила} & \textbf{EMA} & \textbf{WRC} & \textbf{RRC} \\
	\endfirsthead
	\toprule
	\textbf{Правила} & \textbf{EMA} & \textbf{WRC} & \textbf{RRC} \\
	\midrule
	\endhead
	\multicolumn{3}{r}{\footnotesize(Продолжение на следующей странице)}
	\endfoot
	\bottomrule
	\endlastfoot
	\multicolumn{4}{c}{\cellcolor{gray!25}Частично отличающиеся правила} \\
	Стартовые очки &
	30000 &
	30000 &
	\textit{30000}\footnote{Допускается 25000} \\
	\midrule
	Ума &
	15/5 &
	15/5&
	\textit{15/5}\footnote{Допускается также 30/10 или любая другая симметричная ума} \\
	\midrule
	Бросок кубиков &
	одинарный &
	одинарный &
	\textit{одинарный}\footnote{Допускается также двойной бросок кубиков (первым броском дилер определяет разламываемую стену, вторым броском игрок, на чью стену указал дилер, определяет место разлома стены).} \\
	\midrule
	Риичи, оставшиеся на столе после завершения игры &
	уходят победителю &
	теряются &
	уходят победителю \\
	\midrule
	В случае равенства очков победителей риичи на кону &
	делятся поровну &
	- &
	делятся поровну \\
	\midrule
	Акадора &
	нет &
	нет &
	\textit{есть}\footnote{Опционально акадоры могут отсутствовать} \\
	\midrule
	Приоритет объявлений &
	рон > пон/кан > чи &
	рон > пон/кан/чи\footnote{В ином случае по первенству объявления} &
	рон > пон/кан > чи \\
	\midrule
	Атамаханэ &
	нет &
	есть &
	\textit{нет}\footnote{Опционально может вводиться атамаханэ} \\
	\midrule
	Кто получает выплаты за хонбу в случае дабл-/триплрона &
	все &
	- &
	все (если применимо) \\
	\midrule
	Кто получает палочки риичи на кону в случае дабл-/триплрона &
	первый по ходу игры &
	- &
	первый по ходу игры (если применимо) \\
	\midrule
	Абортивные ничьи &
	нет &
	нет &
	\textit{есть, кроме тройного рона}\footnote{Допустимо также иметь все абортивные ничьи либо никаких из них} \\
	\midrule
	Добавление хонбы при абортивной ничьей &
	- &
	- &
	да (если применимо) \\
	\midrule
	Пао на сууканцу &
	нет &
	да &
	нет \\
	\midrule
	Банкротство\footnote{Завершение игры в случае, если кто-то из игроков уходит в минус по очкам} &
	нет &
	нет &
	\textit{нет}\footnote{Опционально может вводиться банкротство} \\
	\midrule
	Якитори &
	нет &
	нет &
	\textit{нет}\footnote{Опционально может вводиться якитори} \\
	\midrule
	Кириаге манган\footnote{Округление рук 4/30 и 3/60 до мангана} &
	нет &
	есть &
	нет \\
	\midrule
	Счетный якуман &
	нет &
	нет &
	\textit{есть} \\
	\midrule
	Сложение якуманов &
	нет &
	есть &
	\textit{есть} \\
	\midrule
	Ограбление закрытого кана при выигрыше в 13 сирот &
	можно &
	нельзя &
	можно \\
	\midrule
	Нагаши манган &
	нет &
	нет &
	\textit{нет}\footnote{Опционально может вводиться нагаши манган} \\
	\midrule
	Ренхо &
	манган &
	манган &
	\textit{манган}\footnote{Опционально ренхо может быть исключено из списка допустимых яку полностью} \\
	\midrule
	Фу за ветер места и раунда &
	4 фу &
	2 фу &
	4 фу \\
	\midrule
	После окончания турнирного таймера &
	доиграть текущую раздачу и сыграть еще одну &
	доиграть текущую раздачу &
	доиграть текущую раздачу и сыграть еще одну\footnote{Опционально: доиграть текущую раздачу + раздача, окончившаяся штрафом чомбо, считается сыгранной.} \\
	
	\multicolumn{4}{c}{\cellcolor{gray!25}Полностью совпадающие правила} \\
	Куитан (открытое тан-яо) &
	\multicolumn{3}{c}{есть} \\
	\midrule
	Атодзуке\footnote{Возможность победы, даже если одно из ожиданий не дает яку} &
	\multicolumn{3}{c}{есть} \\
	\midrule
	Куикаэ\footnote{Сброс тайла, завершающего только что взятый сет} &
	\multicolumn{3}{c}{недопустимо} \\
	\midrule
	Ока &
	\multicolumn{3}{c}{0} \\
	\midrule
	В случае равенства очков ума: &
	\multicolumn{3}{c}{делится поровну} \\
	\midrule
	Ренчан назначается в случае если &
	\multicolumn{3}{c}{дилер темпай} \\
	\midrule
	Выплата за хонбу &
	\multicolumn{3}{c}{300} \\
	\midrule
	Суммарные выплаты в случае ничьей &
	\multicolumn{3}{c}{3000} \\
	\midrule
	Дора &
	\multicolumn{3}{c}{есть} \\
	\midrule
	Урадора &
	\multicolumn{3}{c}{есть} \\
	\midrule
	Кандора &
	\multicolumn{3}{c}{есть} \\
	\midrule
	Кан-урадора &
	\multicolumn{3}{c}{есть} \\
	\midrule
	Выход из временного фуритена &
	\multicolumn{3}{c}{дискард} \\
	\midrule
	Рянхансибари &
	\multicolumn{3}{c}{нет} \\
	\midrule
	Открытие индикатора кандоры при объявлении открытого кана &
	\multicolumn{3}{c}{сразу} \\
	\midrule
	Открытие индикатора кандоры при объявлении закрытого кана &
	\multicolumn{3}{c}{сразу} \\
	\midrule
	Пао на дайсанген &
	\multicolumn{3}{c}{да} \\
	\midrule
	Пао на дайсууши &
	\multicolumn{3}{c}{да} \\
	\midrule
	Секинин барай (ответственность за риншан) &
	\multicolumn{3}{c}{нет} \\
	\midrule
	Выплата хонбы при пао &
	\multicolumn{3}{c}{только набросивший в рон} \\
	\midrule
	Агарияме\footnote{Завершение игры в случае лидерства дилера после последней раздачи}&
	\multicolumn{3}{c}{нет} \\
	\midrule
	Западные раунды &
	\multicolumn{3}{c}{никогда} \\
	\midrule
	Бадзоро\footnote{Базовая добавочная стоимость в ханах на каждую выигрышную руку} &
	\multicolumn{3}{c}{2} \\
	\midrule
	Фуритен риичи &
	\multicolumn{3}{c}{допустимо} \\
	\midrule
	Объявление риичи без возможности следующего взятия из стены &
	\multicolumn{3}{c}{недопустимо} \\
	\midrule
	Закрытый кан после риичи, при условии что интерпретация руки не меняется &
	\multicolumn{3}{c}{допустимо} \\
	\midrule
	Рюиисо требует хотя бы пары зеленых драконов &
	\multicolumn{3}{c}{нет} \\
	\midrule
	Сочетаемость риншана и хайтея &
	\multicolumn{3}{c}{нет} \\
	\midrule
	Формальный темпай (без яку) &
	\multicolumn{3}{c}{допустимо} \\
	\midrule
	Если все тайлы, требуемые для победы, уже лежат в открытых сетах &
	\multicolumn{3}{c}{нотен} \\
	\midrule
	+2 фу за победу по цумо &
	\multicolumn{3}{c}{да, кроме пин-фу} \\
\end{tabularx}

В правилах также могут присутствовать менее значимые отличия, пожалуйста, обращайтесь к полному тексту правил для уточнения.

\subsection{Требования к турнирам для аккредитации}

Для аккредитации турнира в рейтинге RR, турнир должен соответствовать следующим требованиям:

\begin{itemize}
	\item Количество игроков: \textbf{16 и более}.
	\item Количество игр: \textbf{4 и более}.
	\item Турнир должен быть \textbf{открытым}.
	\item Играются \textbf{ханчаны}, а за каждым столом \textbf{четыре} игрока.
	\item Правила и схема рассадок на турнире не должны существенно отклоняться от рекомендаций, описанных в данном регламенте. Список допустимых вариаций правил приведен ниже.
\end{itemize}

Открытый турнир --- турнир, в котором как минимум половина мест доступна для открытой регистрации (доступна любому игроку), либо как минимум половина мест распределяется на основании результатов отборочного тура (квалификации), проводимого перед турниром. Квалификация может быть заменой открытой регистрации. Регистрация на квалификацию должна быть открытой для любого игрока.

Допустимые \textbf{вариации правил} (варианты, соответствующие текущему регламенту, выделены жирным; предпочитайте их по возможности):
\begin{itemize}
	\item Акадоры: \textbf{есть} / нет;
	\item Стартовые очки: \textbf{30000} / 25000;
	\item Ума: предпочтительно \textbf{15/5} либо 30/10. Допускается любая другая симметричная ума.
	\item Нагаши манган: \textbf{нет} / есть;
	\item Банкротство (досрочное прекращение игры при уходе игрока в минус): \textbf{нет} / есть;
	\item Абортивные ничьи: \textbf{есть, кроме тройного рона} / есть, все / нет;
	\item Атамаханэ: \textbf{нет} / есть;
	\item Чомбо: \textbf{-20000 после умы} / обратный манган;
	\item Якитори: \textbf{нет} / есть;
	\item Ренхо: \textbf{манган} / отсутствует.
\end{itemize}

Необычные \textbf{вариации рассадок}, используемые на турнире, могут также послужить поводом для недопуска турнира в рейтинг RR. Мы рекомендуем придерживаться следующих вариантов:
\begin{itemize}
	\item Рассадка на турнире полностью предопределенная. Организаторы заранее готовят рассадку на весь турнир и по ней происходят игры. Предопределенная рассадка позволяет глобально и заранее минимизировать пересечения между игроками, но требует предварительной работы от организаторов по ее формированию согласно количеству игроков.
	\item Рассадка на турнире полностью автоматическая и состоит из следующих этапов последовательно:
	\begin{itemize}
		\item 1 или 2 случайных рассадки подряд. На турнирах среднего размера стоит ограничиться одной случайной рассадкой в самом начале; на больших турнирах допускается играть по случайной рассадке первые две игры;
		\item Швейцарская рассадка на все игры в середине турнира;
		\item 2 или 3 интервальные рассадки в конце турнира. Выбор интервала оставляется на усмотрение организаторов. Для мелких однодневных турниров стоит ограничиться одной интервальной рассадкой в последней игре.
	\end{itemize}
\end{itemize}

\newpage

\section{Заключение}

В данном регламенте мы постарались максимально охватить как все вопросы, регулярно возникающие у начинающих игроков, так и нюансы в процессе проведения турниров и клубных игр. В случае возникновения каких-либо вопросов или уточнений, просим обращаться в рабочую группу по правилам, все вопросы обязательно будут рассмотрены и при необходимости будут внесены правки.

Редакции правил предполагается обновлять раз в год. В случае отсутствия существенных правок, выпуск новой редакции может быть отменен. При проведении турниров просим указывать версию (год выпуска) регламента правил, которые вы собираетесь использовать на турнире. Ожидается, что все судьи турнира ориентируются в регламенте и следят за актуальностью своих знаний. Рабочая группа обязуется обеспечить максимальное публичное распространение новых редакций по мере их выхода.

Спасибо за внимание и ждем вас на турнирах!
