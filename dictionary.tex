\section{Словарь терминов}

Ниже приведена таблица общеупотребительных терминов в алфавитном порядке с японскими написаниями и указанием аналогичных терминов (вариантов). Многие из этих терминов интенсивно используются в обсуждениях внутри сообщества, поэтому знание терминологии является хорошим тоном. Рекомендуем пользоваться основным вариантом каждого термина.

Список не является полным, здесь приведены только те термины, которые используются наиболее часто.

\begin{tabularx}{\linewidth}{|L{0.14\linewidth}|L{0.13\linewidth}|L{0.1\linewidth}|L{0.18\linewidth}|L{0.4\linewidth}|}
	\caption{Терминология} \\
	\toprule
	\textbf{Термин} & \textbf{Варианты} & \textbf{Кандзи} & \textbf{Чтение} & \textbf{Значение} \\
	\midrule
	\endfirsthead
	\toprule
	\textbf{Термин} & \textbf{Варианты} & \textbf{Кандзи} & \textbf{Чтение} & \textbf{Значение} \\
	\midrule
	\endhead
	\midrule
	\multicolumn{5}{r}{\footnotesize(Продолжение на следующей странице)}
	\endfoot
	\bottomrule
	\endlastfoot
	
	Агари & - & \textnihon{和了} & \textnihon{あがり} & Объявление победы в раздаче (рон или цумо). \\
	Агарипай & - & \textnihon{和了牌} & \textnihon{あがりぱい} & Выигрышный тайл (завершающий руку до победной комбинации). \\
	Агарияме & - & \textnihon{和了止め} & \textnihon{あがりやめ} & Правило, согласно которому дилер имеет право завершить игру досрочно в случае лидерства после последней раздачи (оорасу). \\
	Адское ожидание & - & \textnihon{地獄待ち} & \textnihon{じごくまち} & Ожидание на единственный оставшийся тайл в игре (т.е. не сброшенный в дискард и не лежащий в открытых сетах). \\
	Акадора & Красная дора & \textnihon{赤ドラ} & \textnihon{あかどら} & Тайл, окрашенный в красный цвет; является дорой независимо от того, что в индикаторах доры. \\
	Анджун & Закрытое чи & \textnihon{暗順} & \textnihon{あんじゅん} & Закрытый сет из трех последовательных тайлов мастей (закрытое шунцу) \\
	Анкан & Закрытый кан & \textnihon{暗槓} & \textnihon{あんかん} & Объявленный закрытый кан (закрытое канцу) \\
	Анко & Закрытый пон & \textnihon{暗刻} & \textnihon{あんこう} & Закрытый сет из трех одинаковых тайлов (закрытое коцу). Анкан также является анко. \\
	Ари & Есть & \textnihon{有り} & \textnihon{あり} & Слово, обозначающее, что какое-либо правило \textit{применяется} в данных правилах стола. Например, \textit{куитан ари} означает, что комбинация тан-яо \textit{может} быть собрана на открытой руке. \\
	Атама & Пара & \textnihon{頭} & \textnihon{あたま} & Пара, состоящая в выигрышной комбинации (4 сета и пара). \\
	Атамаханэ & - & \textnihon{頭跳ね} & \textnihon{あたまはね} & Правило, согласно которому при объявлении победы с одного и того же сброса более чем одним игроком, приоритет имеет игрок, сидящий первым по направлению хода от игрока, сбросившего тайл. \\
	Атодзуке & - & \textnihon{後付け} & \textnihon{あとづけ} & Правило, разрешающее объявлять победу в случае, если наличие яку в руке зависит от ожидания. Также используется для обозначения ситуации, когда наличие яку зависит от ожидания. \\
	Ветер раунда & Бакадзе & \textnihon{場風} & \textnihon{ばかぜ} & Ветер, указанный на индикаторе первого дилера на текущий момент игры. \\
	Бетаори & - & \textnihon{ベタ降り} & \textnihon{べたおり} & Стратегия защиты, заключающаяся в сносе заведомо безопасных тайлов (например, гэнбуцу, невозможных ожиданий) путем разрушения имеющихся форм в руке. \\
	Гэнбуцу & - & \textnihon{現物} & \textnihon{げんぶつ} & Тайл, находящийся в дискарде игрока. На все гэнбуцу-тайлы у игрока фуритен. \\
	Даматен & - & \textnihon{黙聴} & \textnihon{だまてん} & Скрытый темпай на закрытой руке (без объявления риичи). \\
	Дайминкан & - & \textnihon{大明槓} & \textnihon{だいみんかん} & Открытый кан, объявленный при наличии в руке анко тайлов. \\
	Дискард & Сброс & \textnihon{河} & \textnihon{かわ} & Область на столе, куда сбрасываются ненужные тайлы каждым игроком в свой ход. \\
	Дзянсо & - & \textnihon{雀荘} & \textnihon{じゃんそう} & Общее название заведений для игры в маджонг в Японии. \\
	Дилер & Оя & \textnihon{親} & \textnihon{おや} & Раздающий игрок, игрок восточного ветра в текущей раздаче. \\
	Дора & - & \textnihon{ドラ} & - & Бонусный тайл, дающий +1 хан при победе. \\
	Кабэ & - & \textnihon{壁} & \textnihon{かべ} & Техника защиты, использующая невозможность ожиданий в рянмен, пенчан и канчан при полностью видимых граничных тайлах. \\
	Камича & - & \textnihon{上家} & \textnihon{かみちゃ} & Игрок, сидящий слева. \\
	Канцу & Кан (сет) & \textnihon{槓子} & \textnihon{かんつ} & Четыре одинаковых тайла, объявленные как кан. \\
	Канчан & - & \textnihon{嵌張} & \textnihon{かんちゃん} & Внутреннее ожидание в шунцу, например 3-5 \\
	Каратен & - & \textnihon{空聴} & \textnihon{からてん} & Пустое ожидание. Темпай, когда все завершающие руку тайлы уже вышли в дискард. \\
	Кимеучи & - & \textnihon{決め打ち} & \textnihon{きめうち} & Способ игры, при котором игрок с начала раздачи фиксируется на сборе конкретного яку и сбрасывает все неподходящие тайлы. В подавляющем большинстве случаев кимеучи неприемлемо. \\
	Коцу & Пон (сет) & \textnihon{刻子} & \textnihon{こつ} & Три одинаковых тайла \\
	Куикаэ & - & \textnihon{喰い替え} & \textnihon{くいかえ} & Правило, запрещающее после объявления чи или пона снос такого же тайла, который был взят в чи/пон, а также тайла, которые мог дополнять это чи до взятия. \\
	Куитан & - & \textnihon{喰い断} & \textnihon{くいたん} & Правило, разрешающее сбор тан-яо на открытой руке. \\
	Кюсюкюхай & - & \textnihon{九種複数} & \textnihon{きゅうしゅきゅはい} & Пересдача при наличии в стартовой руке 9 уникальных терминальных и благородных тайлов. \\
	Накасудзи & - & \textnihon{中筋} & \textnihon{なかすじ} & Судзи в середине последовательности (например, в случае 1-23-4-56-7 четверка - накасудзи). \\
	Наси & Нет & \textnihon{無し} & \textnihon{なし} & Слово, обозначающее, что какое-либо правило \textit{НЕ применяется} в данных правилах стола. Противоположно "ари". \\
	Не-дилер & Ко & \textnihon{子} & \textnihon{こ} & Игрок, не являющийся оя (дилером) в текущей раздаче. \\
	Ничья & Рюкёку & \textnihon{流局} & \textnihon{りゅうきょく} & Игровая ситуация, когда никто не объявил победу до исчерпания живой стены. \\
	Нобетан & - & \textnihon{延べ単} & \textnihon{のべたん} & Двойное ожидание в пару (танки), например 3-4-5-6. \\
	Номи & - & \textnihon{ノミ} & - & Буквально "только", слово, обозначающее, что кроме данного яку в руке ничего нет. Обычно применяется для яку на 1-2 хан, например "риичи номи". \\
	Нотен & - & \textnihon{ノーテン} & - & Состояние в конце раздачи, когда рука не ожидает последний тайл для завершения выигрышной комбинации. \\
	Объявление & Наки & \textnihon{鳴き} & \textnihon{なき} & Объявление в смысле самого процесса взятия тайла из чужого дискарда. \\
	Ожидание & - & \textnihon{待ち} & \textnihon{まち} & Список тайлов, дополняющих данную руку до выигрышной. \\
	Ока & - & \textnihon{オカ} & - & Бонус победителю в игре. \\
	Оорасу & - & \textnihon{オーラス} & - & Яп. калька от "all last", последний раунд игры. \\
	Пайфу & Реплей & \textnihon{牌譜} & \textnihon{ぱいふ} & Подробная запись игры с расшифровками. \\
	Пао & - & \textnihon{包} & \textnihon{パオ} & Ответственность за сбор последнего пона для якумана. \\
	Пенчан & - & \textnihon{辺張} & \textnihon{ぺんちゃん} & Краевое ожидание в шунцу, например 1-2. \\
	Ренчан & - & \textnihon{連荘} & \textnihon{れんちゃん} & Дополнительная раздача, назначаемая при победе дилера и при ничьей. \\
	Риншанпай & - & \textnihon{嶺上牌} & \textnihon{りんしゃんぱい} & Тайл, который берется с мертвой стены при объявлении кана. \\
	Рука & Хайпай & \textnihon{配牌} & \textnihon{はいぱい} & Набор тайлов игрока. Вариант "хайпай" чаще (а в японских источниках - всегда) используется для обозначения именно стартовой руки, для руки в целом используется термин "техай" (\textnihon{手牌}). \\
	Рянкан & - & \textnihon{両嵌} & \textnihon{りゃんかん} & Двойной канчан, двойное внутреннее ожидание, например 3-5-7. \\
	Рянмен & - & \textnihon{両面} & \textnihon{りゃんめん} & Двустороннее ожидание в шунцу, например 3-4. \\
	Рянхансибари & - & \textnihon{二飜縛り} & \textnihon{りゃんはんしばり} & Правило, согласно которому после 5й хонбы для победы требуется минимум 2 хан. \\
	Сангенпай & - & \textnihon{三元牌} & \textnihon{さんげんぱい} & Тайлы драконов. \\
	Сакигири & Пуш & \textnihon{先切り} & \textnihon{さきぎり} & Потенциально опасный сброс. \\ 
	Сброс & - & \textnihon{切り} & \textnihon{きり} & Выкладывание тайла в дискард в свой ход. \\ 
	Стена & - & \textnihon{山} & \textnihon{やま} & Последовательность еще не взятых тайлов. \\
	Судзи & - & \textnihon{筋} & \textnihon{すじ} & Дословно "интервал", тайлы мастей с промежутком в два тайла между ними (например, 1-4-7). \\
	Счетные палочки & Тенбо & \textnihon{点棒} & \textnihon{てんぼう} & Палочки, используемые для расчетов внутри игры и подсчета очков. \\
	Сянпон & - & \textnihon{双ポン} & \textnihon{しゃんぽん} & Ожидание в два пона, например 4-4 и 9-9. \\
	Танки & - & \textnihon{単騎} & \textnihon{たんき} & Ожидание в пару. \\
	Тацу & - & \textnihon{塔子} & \textnihon{たーつ} & Два тайла, которым недостает третьего до сета. \\
	Темпай & - & \textnihon{聴牌} & \textnihon{テンパイ} & Состояние ожидания в победу, когда до победы не хватает единственного тайла. \\
	Тоби & Банкротство & \textnihon{トビ} & - & Правило, согласно которому игра завершается досрочно, если кто-то из игроков уходит в минус по очкам. \\
	Тоймен & - & \textnihon{対面} & \textnihon{といめん} & Игрок, сидящий напротив. \\
	Тойцу & Пара & \textnihon{対子} & \textnihon{といつ} & Пара одинаковых тайлов. \\
	Тонпуусен & - & \textnihon{東風戦} & \textnihon{とんぷうせん} & Игра, состоящая только из восточных раундов. \\
	Уке-ире & - & \textnihon{受け入れ} & \textnihon{うけいれ} & Количество и типы тайлов, которые способствуют уменьшению числа шантен в руке. \\
	Фуритен & - & \textnihon{振聴} & \textnihon{ふりてん} & Состояние игрока, когда любой из его выигрышных тайлов находится в его дискарде. Игрок является фуритен даже тогда, когда в дискарде лежит тайл, который не принес бы игроку ни одного яку в руке (в случае атодзуке). \\
	Фурикоми & Наброс (жарг.) & \textnihon{振り込み} & \textnihon{ふりこみ} & Сброс тайла, завершающего победную руку другого игрока. \\
	Фуро & Открытия & \textnihon{副露} & \textnihon{ふうろ} & Вся открытая часть руки игрока. \\
	Хайтейпай & Последнее взятие & \textnihon{海底牌} & \textnihon{はいていぱい} & Последний тайл в живой стене. \\
	Ханчан & - & \textnihon{半荘} & \textnihon{はんちゃん} & Игра, состоящая из восточных и южных раундов. \\
	Хаработе & & & & Форма последовательности с парным тайлом в середине, например 4556. \\
	Хонба & - & \textnihon{本場} & \textnihon{ほんば} & Индикатор ренчана, обычно стоочковвая палочка. \\
	Хотейпай & - & \textnihon{河底牌} & \textnihon{ほうていぱい} & Последний сброс в раздаче. \\
	Цумо & - & \textnihon{自摸} & \textnihon{つも} & 1) Победа взятием со стены; 2) Тайл, взятый со стены. \\
	Цумогири & - & \textnihon{自摸切り} & \textnihon{つもぎり} & Сброс тайла, только что пришедшего со стены. \\
	Чомбо & - & \textnihon{冲合} & \textnihon{ちょんぼ} & Штраф за нарушения в игре, после которых игру невозможно продолжать. \\
	Шимоча & - & \textnihon{下家} & \textnihon{しもちゃ} & Игрок, сидяющий справа. \\
	Шантен & - & \textnihon{向聴} & \textnihon{シャンテン} & Количество тайлов, которое необходимо заменить в руке, чтобы выйти в темпай. \\
	Шонпай & - & \textnihon{生牌} & \textnihon{しょんぱい} & Тайл, еще ни разу не выходивший и не лежащий в открытых сетах. \\
	Шоминкан & - & \textnihon{小明槓} & \textnihon{しょうみんかん} & Открытый кан, полученный докладыванием открытого пона до кана. \\
	Шунцу & Чи (сет) & \textnihon{順子} & \textnihon{しゅんつ} & Три тайла одной масти, образующие последовательность. \\
	Якитори & - & \textnihon{焼き鳥} & \textnihon{やきとり} & 1) Правило, согласно которому игрок, который не одержал ни одной победы в игре, выплачивает штраф. 2) Состояние игрока, который не одержал ни одной победы в игре. \\
\end{tabularx}

Часто используемые жаргонизмы:
\begin{itemize}
	\item \textit{Кинуть палку}, \textit{кинуть риичи} - объявить риичи.
	\item \textit{Наброс} - сброс тайла, на котором объявляется победа, см. фурикоми. Иногда также обозначает сброс тайла, на котором делается объявление сета.
	\item \textit{Очко} - единица масти пин.
	\item \textit{Переехать} - обойти по очкам либо не дать победить в раздаче, реализовать преимущество в игре.
	\item \textit{Петух} - единица масти соу (птичка).
	\item \textit{Русское танки} - ожидание в танки в форме хаработе.
	\item \textit{Русский сянпон} - ожидание в два сянпона, каждый из которых находится в середине рянмена (напр., 45556 + 12223).
	\item \textit{Хиросима} - вариант риичи-маджонга на троих. В японоязычной и англоязычной среде обычно называется "санма".
\end{itemize}
	
Обратите внимание, что данный список приведен здесь только в образовательных целях. В общем случае мы не рекомендуем использовать жаргонизмы в повседневной игре, однако знание их может помочь в разговоре с другими игроками.
