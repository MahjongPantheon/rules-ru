\section{Проведение турниров}

В данном разделе мы опишем лучшие практики организации оффлайн-турниров, которые могут быть использованы в качестве чеклиста для вашего турнира, а также быть руководством для начинающих организаторов.

Обычно подготовка турнира начинается за несколько месяцев до предполагаемой даты. В самом начале организатору предстоит определиться с датой проведения, а для этого в первую очередь необходимо договориться насчет помещения, в котором будет проводиться турнир.

\subsection{Поиск и подготовка помещения}

Обычно подготовкой турнира в городе занимаются жители этого города, поэтому мы предполагаем, что организаторы в курсе потенциальных мест для проведения турниров. Следующие варианты берутся в качестве помещения чаще всего:
\begin{itemize}
	\item Конференц-залы (отдельные или при гостиницах);
	\item Общественные места, предполагающие проведение массовых мероприятий:
	\begin{itemize}
		\item отдельные залы;
		\item залы при муниципальных заведениях;
		\item антикафе;
		\item библиотеки;
		\item заведения общественного питания;
	\end{itemize}
	\item Турниры на свежем воздухе (впрочем, обычно такие турниры делаются для весьма небольшого числа человек).
\end{itemize}

Для начала следует \textit{свериться с расписанием турниров} на ближайшие месяцы и убедиться, что ваш турнир не будет пересекаться по датам с другими, уже анонсированными турнирами. Более того, желательно выбрать дату таким образом, чтобы в диапазоне двух недель в каждую сторону не было анонсировано других турниров --- это нужно для того, чтобы дать игрокам отдохнуть, далеко не каждый готов приезжать на турнир каждые выходные.

Далее необходимо заранее \textit{договориться с местом проведения о точной дате} и о том, что вас случайным образом не попросят перенести мероприятие --- иногородним игрокам будет крайне неудобно менять билеты и бронирования в случае переноса или отмены мероприятия. Идеально, если точная дата будет известна хотя бы за 1 месяц для турниров на 20-50 человек и за 2 месяца на турниры более 50 человек.

Будьте готовы, что некоторые места могут попросить внести частичную или полную \textit{предоплату} за аренду помещения или \textit{залог}.

Помимо собственно зала, следует озаботиться отдельным \textit{помещением для зоны перекуса и зоной отдыха}, в которую могут выйти игроки после завершения игры за их столом. В качестве такой зоны может служить лобби отеля или коридор между залами. Также следует подумать про зону для курения, обычно это уличное пространство перед входом на площадку, но могут быть и другие варианты (балконы, специальные зоны для курения) --- о том, где именно можно курить, следует также явно написать в специальном регламенте турнира.

В случае, если ваш турнир предполагается проводить более одного дня, озаботьтесь договоренностью о том, что турнирное помещение будет \textit{закрыто от посторонних на вечер и ночь} --- вам будет гораздо удобнее оставить столы и наборы в помещении, чем выгружать их каждый вечер и загружать их обратно каждое утро.

Также следует заранее позаботиться о \textit{дополнительном оборудовании} --- например, проекторе и экране (если зал предусматривает их), звуковом оборудовании (микрофон, усилители, громкоговорители). Часто дополнительное оборудование можно взять в аренду в проводимом зале, либо оно может уже иметься на месте проведения. Обратите внимание, что по регламенту требуется, чтобы игроки могли видеть оставшееся до конца игры время в любой момент --- это может быть достигнуто при помощи либо использования электронного ассистента, либо проектора или экранов в зале (при их наличии).

Рекомендуется иметь хотя бы один \textit{ноутбук} для организаторских нужд, обычно его предоставляет кто-либо из организаторов. С этого ноутбука в том числе может транслироваться на проектор таймер и рассадка перед каждой игрой.

\subsection{Расписание, взносы и регламент}

Далее необходимо определиться с \textit{расписанием и регламентом} турнира. Мы предлагаем использовать текущий регламент как базу и стараться минимально отклоняться от правил и рекомендаций. Некоторые существенные отклонения могут послужить причиной невключения турнира в RR-рейтинг, в частности:
\begin{itemize}
	\item Турнир не является открытым для всех игроков. Если вы предполагаете, что участвовать смогут только игроки, прошедшие некоторый отбор, или вводите ограничения по географическому признаку, турнир будет учтен только в рейтинге CRR;
	\item Правила турнира существенно отклоняются от описанных в данном регламенте, например следующие турниры могут быть учтены только в CRR:
	\begin{itemize}
		\item Турнир по маджонгу на троих (санма);
		\item Турнир с существенными отклонениями от правил текущего регламента (в т.ч. вводящий особые правила, которые не являются общепринятыми в сообществе), далее будет приведен список допустимых вариаций;
		\item Турнир, предполагающий денежные выплаты между игроками по итогам игр.
		\item Турниры “на выживание”, длящиеся более 5 ханчанов за день;
		\item Турниры с тонпусенами вместо ханчанов;
	\end{itemize}
\end{itemize}

Также не забудьте ознакомиться со списком критериев для аккредитации турнира в RR-рейтинг, приведенным в приложении. Менее значимые вариации правил следует обсудить при подаче заявки на проведение турнира, однако мы рекомендуем избегать любых нестандартных правил.

Рассадки на турнире также следует заранее обозначить в соответствии с рекомендациями в приложении. Обратите внимание, что предполагаемую схему рассадок на турнире необходимо включить в регламент турнира, чтобы игроки заранее знали что их ждет.

\subsection{Количество игроков}

Количество игроков обычно ограничено не только снизу, но также и сверху, исходя из вместимости площадки проведения турнира. В случае, если на турнир зарегистрировалось больше игроков, чем может принять площадка, у организаторов есть два возможных варианта дальнейших действий:
\begin{itemize}
	\item Попытаться в срочном порядке найти площадку большего размера, при чем крайне желательно, чтобы площадка располагалась недалеко от исходной;
	\item Объявить явно, что игроки, зарегистрировавшиеся после определенного количества игроков в списке, проходят на турнир в порядке живой очереди и могут не попасть на турнир. В этом случае за неделю или две до турнира имеет смысл связаться лично с игроками, которые не попадают на турнир с большой вероятностью, и предупредить их об этом.
\end{itemize}

Хорошим тоном также считается предварительный опрос игроков в личном общении, если организаторам известны личные контакты игроков. В ином случае имеет смысл ориентироваться на то, внес ли игрок взнос за турнир.

\subsubsection{Ранжирование турниров}

Турниры подразделяются на крупные, средние и мелкие (локальные). Базовым принципом для отнесения турнира к той или иной категории служит количество зарегистрировавшихся игроков:
\begin{itemize}
	\item Мелкие и локальные турниры: до 40 человек
	\item Средние турниры: от 40 до 80 человек
	\item Крупные турниры: более 80 человек
\end{itemize}

На мелких и средних турнирах в игрокам предъявляются более мягкие требования, в частности, судья может применять более мягкие санкции за нарушения. На крупных турнирах предполагается, что игрок уже имеет некоторый опыт участия в турнирах и в курсе возможных нарушений и наказаний за них, поэтому поблажки для игроков должны быть сведены к минимуму.

Мелкие и средние турниры должны подчиняться принципу открытости, т.е. в них может принимать участие любой игрок, заявивший о намерении участвовать. На крупных турнирах допускается введение предварительной квалификации на усмотрение организаторов. О наличии квалификации на турнир необходимо оповещать в анонсе турнира.

\subsubsection{Квалификация на турнир}

Предполагается, что любой игрок на турнире обладает базовыми навыками для игры и в курсе текущего турнирного регламента. Организаторы вправе не допустить на крупный турнир игрока, который не подтвердил свои базовые навыки в клубных играх или в более мелких турнирах. Вынесение решения о допуске игрока до турнира осуществляется на базе истории его участия в турнирах (историю можно найти на маджонг-портале). В случае сомнений, организатор вправе запросить у игрока подтверждение его участия в клубных играх --- для этого достаточно предоставить рейтинговую таблицу любого локального клубного рейтинга, в котором игрок сыграл хотя бы 10 игр.

\subsection{Время и расписание}

Для игровых сессий по регламенту предусматривается фиксированное время в \textbf{75 минут} основного времени. По истечении основного времени игроки должны доиграть текущую раздачу и сыграть еще одну.

Примечание: также допускается вариант с доигрыванием текущей раздачи и завершением игры. При этом раздача, завершившаяся с чомбо, считается сыгранной. Предполагается, что интервал основного времени в этом случае будет увеличен до 85-90 минут, поскольку на доигрывание игры будет потрачено меньше времени.

При составлении расписания организаторам имеет смысл рассчитывать общее время каждой игры в 90 минут, завершение оставшихся игр за 15 минут обычно не является проблемой для игроков. В случае задержек может иметь место корректирование расписания, о чем следует предупредить всех игроков.

Обычно предполагается, что турнир начинается рано утром (9 или 10 часов утра --- оптимальное время начала). Имейте в виду, что игроки будут подходить на турнир в течение какого-то времени и закладывайте это время на сбор людей. В первый день обычно предполагаются короткие вступительные мероприятия, это тоже следует учесть в расписании. В последний день турнира организаторы часто стремятся отпустить игроков пораньше, чтобы иногородние игроки успели попасть на рейсы, поэтому часто в последний день делается меньше игр, чем в предыдущие.

В зависимости от организации обеденных перерывов, игрокам нужно дать больше или меньше времени на обед, но в среднем стараются делать обеденный перерыв хотя бы на 1 час.

Пример хорошего расписания турнира:

\textbf{День 1}

\begin{itemize}
	\item \textit{9:00} --- сбор игроков
	\item \textit{9:50} --- вступительное слово организаторов
	\item \textit{10:00} --- начало первой игры
	\item \textit{11:30} --- перерыв
	\item \textit{11:45} --- начало второй игры
	\item \textit{13:15} --- перерыв на обед
	\item \textit{14:30} --- начало третьей игры
	\item \textit{16:00} --- перерыв
	\item \textit{16:15} --- начало четвертой игры
	\item \textit{17:45} --- перерыв
	\item \textit{18:00} --- начало пятой игры
	\item \textit{19:30} --- окончание первого дня турнира, консервация зала
	\item \textit{20:00 и до ночи} --- междудневные мероприятия
\end{itemize}

\textbf{День 2}

\begin{itemize}
	\item \textit{9:30} --- сбор игроков
	\item \textit{10:00} --- начало шестой игры
	\item \textit{11:30} --- перерыв
	\item \textit{11:45} --- начало седьмой игры
	\item \textit{13:15} --- перерыв на обед
	\item \textit{14:30} --- начало восьмой игры
	\item \textit{16:00} --- перерыв
	\item \textit{16:15} --- начало девятой игры
	\item \textit{17:45} --- подведение итогов, подготовка призов
	\item \textit{18:15} --- награждение призеров
	\item \textit{18:45} --- окончание второго дня турнира, уборка в зале, сбор столов и наборов
	\item \textit{19:30 и до ночи} --- послетурнирные мероприятия
\end{itemize}

Обратите внимание, что данное расписание является ориентировочным, но его необходимо опубликовать в анонсе турнира.

\textit{Задержки и отклонения от расписания} в последний день турнира являются более критичными, чем во все остальные дни, поскольку у иногородних участников с большой вероятностью уже куплены билеты на рейсы. Обратите внимание на то, что перерывы в примере длятся по 15 минут --- некоторое время от перерывов может быть использовано для компенсации задержек, очень рекомендуем предусматривать такую буферную зону.

\textbf{Размеры взносов} за турнир следует рассчитывать из следующих соображений по затратам:
\begin{itemize}
	\item Арендная плата за помещение (обычно следует ориентироваться на плату в размере не менее половины от суммы ожидаемых взносов);
	\item Арендная плата за дополнительное оборудование, если оно предполагается;
	\item Стоимость призов и приятных дополнений, если они предполагаются;
	\item Стоимость снеков, чая и прочих междуигровых перекусов;
	\item Стоимость транспортных услуг, в случае если на место нужно доставить столы;
	\item Поощрения для организаторов.
\end{itemize}

Последний пункт является опциональным, но мы рекомендуем его закладывать в стоимость, поскольку часто образуются дополнительные издержки, и в случае их отсутствия лучше распределить остатки средств по организаторам, чем распределять по организаторам же эти дополнительные издержки.

\subsection{Поиск наборов, столов, судейского состава, игроков замены}

\textbf{Поиск наборов и матов} может стать существенной проблемой для больших турниров, поэтому нужно заранее попросить игроков принести свои наборы. Учет наборов ложится на организатора, однако в любом случае лучше иметь пару запасных наборов.

Организаторы турнира также принимают на себя ответственность за возврат наборов и матов владельцам в целости. Рекомендуется вести учет --- какой набор на каком столе разложен. Хорошей практикой является вкладывание в коробку набора и в оболочку для мата номер стола, на котором этот набор используется --- это поможет при сборе наборов после турнира. Нельзя допускать, чтобы наборы перепутались (например, набор положили в коробку от другого набора; такие случаи имели место).

Использование акадор в наборах является обязательным, если иное не указано в сопутствующем регламенте конкретного турнира из-за обстоятельств непреодолимой силы (например, в случае, если у организаторов недостаточно наборов с акадорами).

На используемые на турнире наборы накладывается ряд ограничений, которые также должны учитываться организаторами при подготовке турнира:

\begin{itemize}
	\item Предельный размер тайла --- 34 мм в высоту. Тайлы большего размера с большой вероятностью не поместятся на стол, либо игрокам будет неудобно строить стены и играть таким набором;
	\item Цвет основного тела --- белый или светло-бежевый. Тайлы с черным телом не допускаются на турниры, кроме самых крайних случаев (например, мелкие турниры, на которых нет возможности обеспечить наборы в достаточном количестве). Цвет рубашки не регламентируется, однако рекомендуется, чтобы рубашка по цвету отличалась от основного тела;
	\item Степень изношенности набора --- слишком старые наборы, имеющие повреждения или царапины, позволяющие идентифицировать конкретные тайлы, не следует допускать на турниры;
	\item Наборы с нестандартной рисовкой допускаются к применению на турнирах, при условии что любой взятый тайл в наборе возможно однозначно идентифицировать, не глядя на все остальные тайлы в наборе.
\end{itemize}

В случае, если на турнире используются наборы для автостола, следует оповестить об этом игроков (чтобы игроки не пытались класть взятые тайлы сверху руки - это может привести к тому, что тайл будет развернут лицевой стороной к соперникам из-за магнитов внутри тайлов).

Организаторы также отвечают за подготовку подходящих для игры \textbf{квадратных столов и достаточного количества стульев}. Если площадка проведения может предложить квадратные столы и стулья --- поздравляем, у вас только что стало чуть меньше проблем. В случае, если квадратные столы необходимо доставить на площадку проведения, следует также заранее озаботиться о транспортировке и инструментах для сбора и разбора столов (и лучше если инструментов будет несколько, для ускорения процесса).

\textbf{Судейский состав} на турнире часто полностью или частично совпадает с организаторским, однако мы рекомендуем заранее озаботиться привлечением судей со следующим расчетом:
\begin{itemize}
	\item В случае если на турнире участвует 80 и более человек, должен быть минимум один неиграющий судья;
	\item Количество играющих судей определяется исходя из расчета одного судьи на 4-5 столов.
\end{itemize}

Начинающие судьи без судейского опыта должны при необходимости опираться на судейство других судей, нежелательно наличие в судейском составе более одного начинающего судьи.

\textbf{Игроки замены} на турнир также должны быть найдены заранее из местных игроков. Игроки замены делятся на постоянных и эпизодических. 

Постоянный игрок замены при необходимости садится за стол в первый день и играет до конца турнира. На первый день следует найти не менее трех постоянных игроков замены, чтобы можно было добить столы до кратного четырем числа игроков. В случае, если не получилось найти троих игроков, но удалось найти двух, один человек из организаторского состава должен быть готов отказаться играть в турнире в первый день, поскольку организаторы не вправе просить кого-либо из игроков сняться с турнира.

Эпизодические игроки замены должны присутствовать в качестве наблюдателей на турнире и быть готовы подменить игрока, который в какой-то момент не смог продолжать игру. Обратите внимание, что если игрок не уверен, сможет ли он присутствовать в первый день турнира, он может сыграть в турнире только в качестве эпизодического игрока замены в последующие дни. 

В общем рейтинге постоянные игроки замены учитываться не должны. Для эпизодических замен предусматривается фиксированный штраф в -30000 очков независимо от занятого места (обычно ставится -15000 штрафа и -15000 умы, в случае иной умы штраф для игроков замены должен быть оговорен отдельно в специальном регламенте турнира).

Организаторы отвечают за корректность состава наборов за всеми столами. Если в процессе игры выясняется, что комплект тайлов не соответствует стандарту (например, не извлечены лишние тайлы пятерок, либо извлечены неверные тайлы и пятерок какой-то масти стало пять), ситуацию нужно исправить немедленно. Штрафов для игроков в этом случае не следует, поскольку в ситуации нет вины игроков. Отмены раздач и переигрывания также не происходит, однако в случае, если текущая раздача не может продолжаться нормальным образом, судья может вынести решение о перераздаче с добавлением дополнительного времени столу.

Рекомендуется иметь на каждом столе распечатанную шпаргалку со списком яку и таблицей подсчета очков. Подготовка шпаргалок возлагается на организаторов турнира.

\subsection{Подготовка призов и приятных дополнений}

Часто на турнирах предусматриваются ценные призы --- они должны быть сделаны/закуплены заранее. Не следует выдавать деньги в конверте в качестве приза --- в ином случае организаторы рискуют подпасть под статью о незаконной организации и проведении азартных игр. Наиболее частые варианты призов:
\begin{itemize}
	\item Кубок и/или грамота
	\item Памятная сумка-шоппер с принтом
	\item Памятная кружка с принтом
\end{itemize}

Следует с осторожностью относиться к ценным призам, которые могут подойти не всем. Футболка с принтом скорее всего не подойдет по размеру, персональная техника может не понравиться, подарочные сертификаты в местные заведения точно не подойдут иногородним игрокам. Другими словами, основное правило --- обязательно подумать, какой приз мог бы подойти любому победителю, кем бы он ни оказался.

Иногда на турнирах также встречаются приятные дополнения для всех игроков --- например, организаторы могут выдавать каждому участнику памятные наклейки, магнитики или иные мелочи. Такие вещи также следует включить в общий бюджет турнира и озаботиться их заказом заранее.

\subsection{Освещение турнира в публичном поле и взаимодействие с игроками}

Очевидно, что чтобы на турнир пришли игроки, они должны в первую очередь об этом турнире узнать тем или иным образом. Основные способы донести анонс:
\begin{itemize}
	\item Публикация на маджонг-портале
	\item Публикация в чатах и каналах
	\item Очная реклама на других турнирах
	\item Личные приглашения
\end{itemize}

Анонс на портале должен включать \textbf{даты турнира, расписание, размеры взноса, локацию, аккредитацию (RR/CRR), контакты организаторов}. Если площадка предъявляет какие-либо требования к игрокам (например, иметь сменную обувь), или имеет какие-то особенности (например, на площадке может быть холодно и нужно запастись теплыми вещами), об этом также следует написать в анонсе на портале. Прочие публикации стоит делать более краткими, но давать ссылку на анонс на портале для уточнения деталей. В отличие от анонса на портале, анонсы в других источниках имеет смысл повторить несколько раз для привлечения как можно большего объема аудитории.

Полезно будет оставить в анонсе инструкцию и схему того, как добираться до локации от ближайших транспортных узлов (метро, остановок). Иногда может быть полезно также заснять видео прохода до площадки, если путь нетривиален.

Когда игроки пришли на турнир, их следует держать в курсе происходящего при помощи анонсов перед играми, а также посредством \textit{специально созданного чата}.

Рекомендуем создать \textbf{специальный чат турнира}, в котором будут публиковаться важные сведения как до турнира, так и в процессе и после него. Кроме того, такой чат поможет игрокам организоваться на междудневные и послетурнирные мероприятия, независимо от того, предполагается ли их централизованная организация или нет. В этом же чате можно будет решать срочные и животрепещущие вопросы с игроками.

Рекомендуется публиковать (и, при возможности, закреплять) следующую информацию в чатах:
\begin{itemize}
	\item До турнира:
	\begin{itemize}
		\item Любые непредвиденные изменения в регламенте, расписании, местоположении;
		\item Напоминания о том, что турнир близится (за месяц и за неделю до старта);
		\item (опционально) Список предлагаемых мест для обеда недалеко от места проведения;
		\item (опционально) Список гостиниц/хостелов, где можно остановиться иногородним.
	\end{itemize}
	\item Во время турнира:
	\begin{itemize}
		\item Обязательно следует закреплять время начала следующей игры. Это особенно важно в случае, если произошел сдвиг в расписании.
		\item По окончании дня, если предусмотрена централизованная организация досуговых мероприятий, закрепить место и время, в котором ждут игроков на ужин и внетурнирные игры.
	\end{itemize}
\end{itemize}

Модерация чата оставляется на усмотрение организаторов, однако общая рекомендация --- не допускать проявлений неуважения, неуместного контента, дискриминирующих и унижающих достоинство высказываний. После завершения турнира и после того, как все игроки выскажут свои впечатления и отзывы в чате, имеет смысл закрыть возможность писать сообщения в чат. 

Многие клубы имеют единственный турнирный чат, который используется периодически во время проведения турниров в городе. Другим вариантом может быть создание отдельных чатов под каждый турнир. Решение, какой именно вариант использовать, оставляется на организаторов. Пожалуйста, не удаляйте чаты после завершения турнира.

\subsection{Обеды и афтерпати}

Организаторы могут предложить централизованный обед для игроков, если площадка такое позволяет (например, при площадке есть ресторан или столовая). Договоренность с заведением питания нужно заключить \textit{заранее}, получив от заведения примерное меню и стоимость.

Для сбора сведений о тех, кто собирается обедать организованно, следует использовать интерактивные формы (Google Forms или аналогичные). В опросе нужно указать:
\begin{itemize}
	\item Ориентировочную стоимость обеда;
	\item Варианты блюд (если заведение питания предлагает такие варианты), либо обозначить конкретные блюда, которые будут на обеде, либо обозначить, что игроки смогут выбрать блюда на месте;
	\item Возможные аллергии и предпочтения (например, вегетарианская кухня).
\end{itemize}

Идеальным вариантом будет пофамильный список людей и выбранных блюд, который нужно предоставить в заведение питания за несколько дней до турнира. Имейте в виду, что чтобы накормить 50 и более человек одновременно заведению необходимо некоторое время, поэтому в договоренность с заведением обязательно следует включить точное время, к которому начнут приходить игроки на обед, чтобы заведение успело подготовить блюда к приходу игроков. Пожалуйста, убедитесь, что в заведении точно приняли к сведению, что в определенный день в определенное время ожидается большое количество посетителей --- это позволит свести к минимуму возможные опоздания после обеда.

В случае, если организаторы не планируют организовывать обеды, следует предложить игрокам как минимум несколько \textit{вариантов, где можно перекусить}. Если вокруг места проведения имеется только одно или два заведения питания, имеет смысл заранее сходить к ним и оповестить о том, что ожидается большое количество посетителей в определенные дни и в определенное время. Также в таком случае будет нелишним попытаться организовать \textit{отдельную комнату} для обеда на площадке проведения, чтобы там могли перекусить те, кто решит заказать персональную доставку блюд.

Организация междудневных досуговых активностей и афтерпати после турнира является опциональной, однако часто организаторы имеют такую возможность. При выборе помещения имеет смысл заранее поинтересоваться, кто из игроков планирует пить/есть/играть после игровых дней, и исходя из этого выбирать заведение. Желательно, чтобы в заведении имелись круглые или квадратные столы, на которых поместились бы игровые маты. Прямоугольные столы также часто приемлемы, хоть и не очень удобны.

С заведением следует договориться \textit{заранее}. Локацию и время, с которого игроков будут ожидать в заведении, следует \textit{закрепить в чате турнира}.

Поинтересуйтесь в заведении, возможно ли будет разделить счет между игроками. Некоторые заведения настаивают на том, чтобы счет был один на всех, это может быть поводом для выбора другого заведения.

\subsection{Междуигровые перекусы}

Хорошей практикой является наличие \textit{чайной зоны}, где можно налить чай и съесть пару печенек. Если при площадке проведения имеется кафе, это может быть хорошим вариантом. В ином случае организаторам придется озаботиться закупкой чая, сахара, салфеток, одноразовой посуды и воды (если воду не предоставляет площадка). Наличие чайника также будет нелишним, поскольку даже если площадка предоставляет кулер с водой, горячая вода в кулере в перерыве иссякнет очень быстро.

Доведите до сведения игроков, что \textbf{приносить еду и питье за игровые столы недопустимо} (в виде исключения можно позволить плотно закрывающиеся емкости с водой или иным напитком).

Любая алкогольная продукция во время турнира должна быть \textbf{под строгим запретом}. Игроки в состоянии алкогольного опьянения могут быть немедленно дисквалифицированы с турнира.

\subsection{Непосредственно перед турниром}

Организаторам следует позаботиться о том, чтобы к началу игр в каждый день \textit{все столы были полностью готовы} --- имели маты, наборы, индикаторы, наборы палочек в требуемом количестве. 

Расстановка столов в зале должна отвечать следующим требованиям:
\begin{itemize}
	\item Столы и стулья не должны мешать свободному проходу судей;
	\item Судья должен иметь возможность подойти к любому столу для решения вопросов;
	\item Два соседних по номеру стола должны стоять рядом, чтобы игроки могли пройти по цепочке до нужного стола. Хороший вариант --- расстановка "змейкой";
	\item Первые столы не должны стоять рядом с последними, т.к. за столами в нижней части таблицы чаще происходит шум, который может отвлекать игроков за первыми столами. Это требование может быть ослаблено или вообще отменено в случае небольших турниров (до 8 столов).
\end{itemize}
Помните, что слишком плотная посадка неудобна как для судей, так и для игроков. В отдельных случаях (например, в случае нетривиальной геометрии зала) можно заранее подготовить карту рассадки, чтобы игроки могли сориентироваться по ней. На каждом столе должен иметься хотя бы один номер стола, напечатанный или нарисованный достаточно крупными цифрами, чтобы быть видным издалека. Рекомендуется размещать номера столов очевидным образом, чтобы игроки не искали свой стол слишком долго.

Имеет смысл очертить в зале небольшую зону для складирования личных вещей (рюкзаков, сумок, верхней одежды). Это позволит игрокам иметь минимум необходимых вещей во время игры вокруг стола и расчистит проходы между столами.

Перед стартом турнира следует проверить, что всё необходимое оборудование работает штатно, в частности следует проверить:
\begin{itemize}
	\item WiFi в зале;
	\item Проектор (в случае его наличия);
	\item Акустическое оборудование (микрофоны, громкоговорители, в случае наличия);
	\item Компьютеры организаторов;
	\item Электронные системы подсчета очков;
	\item Гонг (в случае наличия).
\end{itemize}

\subsection{Проведение турнира}

Перед стартом первой игры организаторы обычно выступают со вступительным словом. Необходимо заложить на это хотя бы пять минут времени в расписании.

Игра начинается с ударом гонга. Если гонг отсутствует, организаторы должны объявить о начале каждой игры голосом в микрофон.

После старта игр организаторы и судейский состав должны следить за порядком в зале и соблюдением относительной тишины. За слишком громкие выкрики и разговоры судьи вправе выносить вежливые указания и назначать штрафы. После завершения игры все игроки должны выйти из зала, чтобы не мешать другим --- за этим также следует следить.

Вызов судьи во время игры происходит путем поднятия руки. Не следует кричать через весь зал --- это отвлекает других игроков.

Если на турнире ведется фото- или видеосъемка, фотографу или оператору следует заручиться согласием игроков на проведение съемки. Если кто-либо из игроков настаивает на том, чтобы его фотографии или сцены видео с его участием были удалены, фотографу или оператору следует прислушаться к просьбе игрока.

Если на турнире присутствуют сторонние наблюдатели, им \textbf{запрещено} любым образом взаимодействовать с игроками за столом (в том числе любым способом показывать любые эмоции --- противники могут использовать такую информацию для оценки силы руки игрока). В случае нарушения наблюдатель может быть удален из зала судейским составом. Некоторые игроки принципиально не допускают, чтобы кто-либо наблюдал за их рукой --- наблюдатели должны уважать это желание и переместиться в другое место по первой же просьбе игрока.

Между играми зал следует обязательно \textit{проветривать}. Как вариант, если в зале предусмотрена система кондиционирования, можно держать ее включенной во время турнира. Если турнир проводится в холодное время года и для проветривания необходимо открывать окна, следует предупредить игроков о том, чтобы они озаботились теплыми вещами.

На время подведения результатов турниров, рекомендуется \textit{удалить игроков из зала}. О точном времени начала награждения следует сообщить в турнирном чате и закрепить сообщение. После награждения может быть организовано общее фото, об этом также следует заранее сообщить игрокам (например, в устном анонсе перед началом последней игры).

\subsection{После турнира}

В течение нескольких дней после завершения турнира организаторам необходимо отправить на портал список игроков согласно занятым местам, а также собрать с судейского состава протоколы нарушений и предоставить их в судейскую коллегию.

Подача итоговых результатов в общие рейтинги осуществляется не позднее, чем через неделю после окончания турнира. В случае, если в итоговом рейтинге турнира у игроков равное количество набранных очков, считается, что игроки делят место между собой.

Если во время турнира работал фотограф, после получения фотографий их имеет смысл опубликовать в турнирном чате.

После турнира игроки могут делиться обратной связью с организаторами в турнирном чате --- эту обратную связь не следует игнорировать. Из всего описанного игроками следует извлечь следующие пункты:
\begin{itemize}
	\item Что понравилось на турнире, на афтерпати, на междудневных играх? На последующих турнирах следует повторить то, что получилось хорошо.
	\item Что не понравилось? На последующих турнирах следует приложить усилия, чтобы исправить плохие моменты.
	\item Конкретные предложения игроков по организации. Их следует записать, оценить и попытаться реализовать в следующий раз.
\end{itemize}

Также имеет смысл поделиться с другими организаторами как хорошими практиками, так и совершенными ошибками и планируемыми действиями по их исправлению. 