\section{Организационная структура комитетов}

В данном регламенте предполагается относительно большой уровень самостоятельности игроков, организаторов, судейского состава и иных лиц, принимающих решения, однако в тех случаях, когда требуется взаимодействие между различными сторонами, предлагается использовать организационные структуры и способы вынесения решений, описанные ниже.

\subsection{Оргструктуры}

Сообщество игроков состоит из, собственно, игроков, некоторые из которых наделяются рядом расширенных прав в плане принятия решений. Выделяются следующие структуры:

\begin{itemize}
	\item \textbf{Организационный комитет} --- группа, состоящая из действующих организаторов турниров в стране, на которую возлагается согласование дат турниров (с целью равномерного покрытия по месяцам и избежания пересечений по датам), а также работа с дисквалификациями.
	\item \textbf{Судейская коллегия} --- группа, состоящая из аккредитованных судей, решающая вопросы аккредитаций, отзыва аккредитаций, а также обрабатывающая жалобы и апелляции после турниров;
	\item \textbf{Комитет по правилам} --- инициативная группа, принимающая решения относительно изменений в текущем регламенте;
\end{itemize}

\subsection{Организационный комитет}

В организационный комитет входят организаторы всех турниров в стране. Организатором может стать любое инициативное лицо. 

Для координации организационного комитета используется общий чат. Начинающему организатору, который еще не участвует в комитете, следует связаться с любым действующим организатором турниров и попросить добавить его в общий чат.

Организационный комитет отвечает за согласование дат турниров таким образом, чтобы несколько турниров не выпадало на одни и те же даты. Допускается размещение на одних и тех же датах малых и средних турниров, которые предполагают участие в основном местных игроков (например, допускается поставить на одни и те же даты турниры в Иваново и во Владивостоке). Крупные турниры (с ожиданием 100 и более игроков) не должны пересекаться по датам, и кроме того интервал между двумя крупными турнирами должен составлять по меньшей мере 4 недели.

Предполагается активный обмен опытом между членами организационного комитета в области организации турниров (особенно в случае появления новых неопытных организаторов). Многие моменты касаемо организации уже описаны в текущем регламенте, однако если возникают общие проблемы, которые еще не описаны, имеет смысл внести эти проблемы и их возможные решения в раздел об организации турниров --- в этом случае организационный комитет должен сформировать предложение о правках в регламенте и предоставить его в комитет по правилам.

В случае, если в организационный комитет поступают жалобы на организацию турнира, комитет обязан рассмотреть жалобы и сформировать предложения по их решению, и довести эти предложения до всех членов комитета.

\subsubsection{Недопуск игрока на турнир}

К игроку, дисквалифицированному с одного турнира, могут быть применены жесткие санкции в виде запрета на участие в любых рейтинговых турнирах в течение определенного времени. Решение о недопуске игрока на определенный срок принимается также организационным комитетом на основании жалоб от игроков и заявлений от судейских коллегий.

Все члены организационного комитета обязуются \textbf{не принимать} на свои турниры игроков, которые не допущены до участия на момент проведения турнира.

Решение о недопуске на основании жалоб от игроков следует выносить осторожно и принимать во внимание все обстоятельства. Решение о недопуске на основании судейского заявления предполагает, что у судьи есть сильные доказательства нечестной игры, которые могут быть подтверждены как минимум одним игроком, не входящим в число аккредитованных судей.

В отдельных случаях, может возникнуть ситуация, когда игрок так или иначе совершил чистосердечное признание в том, что играл нечестно. При предъявлении соответствующих ссылок или свидетельств игроков, решение о недопуске не подлежит дальнейшему расследованию независимо от того, насколько правдивым является чистосердечное признание, т.е. шутники будут наказываться настолько же строго, насколько и нарушители.

\subsubsection{Наблюдатели}

Любой игрок, участвующий в турнире, может добровольно стать наблюдателем турнира и фиксировать любые нарушения со стороны судей и организаторов. По итогу турнира наблюдатель формирует отчет, который следует рассмотреть в организационном комитете и в общей коллегии судей. 

Также на основании отзыва от наблюдателя судейским комитетом может быть принято решение об отзыве аккредитации судьи (см. ниже).

\subsection{Судейская коллегия}

Коллегия --- инициативная группа, занимающаяся принятием решений по поводу поступивших запросов. Судейская коллегия может быть двух видов:

\begin{itemize}
	\item Общая коллегия --- включает в себя всех действующих аккредитованных судей. Занимается рассмотрением жалоб на действия конкретного судьи и имеет право отозвать аккредитацию судьи после рассмотрения жалоб (см. далее);
	\item Рабочая коллегия --- выбирается из числа действующих аккредитованных судей для рассмотрения апелляций игроков на принятые решения в ходе конкретного турнира. Рекомендуется, чтобы число судей в рабочей коллегии было нечетным для обеспечения принятия решений большинством голосов. В рабочую коллегию не должны входить судьи текущего турнира, однако имеет смысл держать их в курсе обсуждения для уточнения мотивации принятых решений.
\end{itemize}

\subsubsection{Порядок получения и обработки апелляций}

В случае, если игрок не согласен с решением судьи, либо к поведению судьи есть претензии, игрок вправе оставить апелляцию. Апелляции принимаются строго после окончания турнира.

В тексте апелляции следует максимально подробно описать спорную ситуацию, вынесенное судьей решение и причины, по которым игрок не согласен с вынесенным решением. Апелляции с недостаточно подробным описанием к рассмотрению не принимаются.

По итогу рассмотренной апелляции игрок имеет право на изменение итоговых рейтинговых очков в турнире, если судейская коллегия решит, что вынесенное решение не соответствует регламенту (например, если судья выписал штраф чомбо вместо назначения мертвой руки). Размер компенсации определяется судейской коллегией. 

Не принимаются апелляции, касающиеся турниров, для которых не применялся данный регламент. Кроме того, устанавливается срок давности в два месяца для рассмотрения апелляций, таким образом апелляции по турнирам, проведенным более чем два месяца назад, также не принимаются.

Рассмотрение апелляций может занимать некоторое время, однако рекомендуется рассмотреть все апелляции в течение месяца после подачи. 

После рассмотрения апелляции, судейская коллегия обязуется сообщить игроку, подавшему апелляцию, о принятом решении.

В случае, если апелляция касается организации турнира, судейская коллегия обязана вынести ее на обсуждение в организационный комитет с целью недопущения повторения ситуации на последующих турнирах.

\subsubsection{Получение статуса судьи}

Судьи турниров обязуются получить аккредитацию на выполнение судейских обязанностей на турнирах. Аккредитация проводится любым из уже аккредитованных судей. В процессе аккредитации судья делает презентацию по данному разделу, а также проводит тестирование претендентов.

Тестирование проводится в форме ряда вопросов с открытыми ответами. По результатам тестирования в получении статуса судьи может быть отказано в случае, если некорректные ответы даны на более чем треть вопросов. Список действующих аккредитованных судей приводится на сайте сообщества риичи в РФ.

Судейский семинар может проводиться как в формате очной встречи, так и онлайн. В случае очной встречи, презентация проводится очно и сразу после презентации следует тестирование. В случае онлайн-аккредитации, претенденту предлагается просмотреть запись презентации и пройти онлайн-тестирование с дальнейшей беседой по удаленной связи. 

\subsubsection{Отзыв аккредитации судьи}

В случае, если к судье возникает слишком много вопросов, статус судьи может быть отозван общей судейской коллегией. Порядок рассмотрения запросов на отзыв статуса судьи:

\begin{itemize}
	\item Получение и фиксирование жалоб на конкретного судью.
	\item Вынесение жалоб на рассмотрение судейской коллегии. Коллегия выносит решение о том, насколько критичны и насколько обоснованы поступившие жалобы. На данном этапе процесс может быть прекращен в случае недостаточной обоснованности или критичности жалоб.
	\item Оповещение судьи о том, какие действия ему следует предпринять, чтобы исправить проблемы, на которые указали игроки.
	\item В случае, если судья не прислушался к замечаниям, наблюдатель ставит на обсуждение вопрос об отзыве судейской аккредитации.
	\item Судейская коллегия рассматривает отзыв наблюдателя и принимает решение об отзыве судейской аккредитации.
\end{itemize}

\subsection{Комитет по правилам}

Данный комитет формируется из заинтересованных лиц, имеющих конкретные предложения по поводу модификации текущего регламента.

В случае, если требуются изменения в регламенте, определяется следующий порядок внесения изменений:
\begin{itemize}
	\item Определение конкретных пунктов, в которых требуются правки ("что правим") и определение конкретных предложений ("как правим");
	\item Предварительный сбор мнений с игроков. Для рассмотрения предложения требуется, чтобы не менее 10 игроков поддержали предлагаемые изменения.
	\item Вынесение предложения на обсуждение в сообществе. Круг заинтересованных лиц не регламентируется и принимаются доводы от любого игрока. В интересах комитета по правилам согласовать изменения с максимально возможным количеством заинтересованных лиц в сообществе, чтобы избежать недопонимания и обеспечить легитимность изменений.
	\item Внесение изменений в регламент и дальнейшая его публикация в соответствии с графиком.
\end{itemize}

Публикация регламента осуществляется \textbf{раз в полгода} и включает все изменения, внесенные за последние полгода. В случае отсутствия изменений, публикация может быть отложена до следующего периода. До публикации изменений следует пользоваться текущей версией регламента, однако допускается точечное внесение грядущих изменений в регламенты турниров организаторами.

Каждое изменение в регламенте должно сопровождаться кратким списком того, что именно было изменено, чтобы людям не приходилось перечитывать регламент полностью каждый раз. В списке изменений следует указывать конкретный номер пункта регламента, который был изменен, а также приводить описание изменения.

В случае, если организуется новый комитет по правилам, настоятельно рекомендуется включать в его состав людей, принимавших ключевые решения при подготовке предыдущих версий регламента. Эти люди смогут объяснить причины появления тех или иных пунктов в текущем регламенте и таким образом избежать повторных длительных обсуждений.
