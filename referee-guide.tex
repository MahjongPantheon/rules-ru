\section{Регламент судейства на турнире}

Основополагающие принципы поведения на любом турнире --- \textbf{вежливость и взаимное уважение}. Однако, все же иногда могут возникать спорные ситуации, для разрешения которых необходимо привлечь независимую сторону --- турнирного судью.

Предполагается, что игроки на турнире:
\begin{itemize}
	\item Дружелюбны;
	\item Придерживаются принципов справедливой игры (fair play);
	\item Честны;
	\item Действуют в духе соревнования;
	\item Стараются действовать таким образом, чтобы неоднозначных и спорных ситуаций не могло возникнуть в принципе;
	\item Воспринимают судей как помощников на случай, если спорная ситуация все же возникла.
\end{itemize}

Так или иначе, иногда возникает необходимость вызвать судью для разрешения спорной ситуации. В этом случае состояние стола должно быть \textbf{зафиксировано до момента вынесения решения судьей}. Замешивание тайлов в любом виде, препятствующее вынесению судейского решения, наказывается произвольным штрафом на усмотрение судьи либо для игрока, осуществившего замешивание, либо для всего стола, если установить конкретного игрока не представляется возможным.

\subsection{Судейские роли}

Главный судья:
\begin{itemize}
	\item Отвечает за координацию работы судейской комиссии;
	\item Отвечает за обеспечение последовательности принятых решений;
	\item Является окончательной инстанцией при разногласиях внутри судейской коллегии;
	\item Осуществляет взаимодействие с организаторами турнира;
	\item Может также исполнять функции судьи.
\end{itemize}

Судья:
\begin{itemize}
	\item Отвечает за судейство на не более чем 10 столах турнира;
	\item Выносит решение максимально быстро, чтобы задержка оказалась минимальной;
	\item Не вступает в споры и не поддается на провокации;
	\item В случае несогласия игрока с решением судьи, судья вправе попросить не спорить и продолжать игру (принцип "замолчи и играй");
	\item Должен брать на заметку игроков с проблемным поведением и делиться этой информацией с другими судьями.
\end{itemize}

Пример возможного развития дискуссии:
\begin{itemize}
	\item Игрок зовет судью.
	\item Судья беседует с игроками и выносит решение.
	\item Игрок недоволен решением и высказывает жалобу судье.
	\item Судья заново обдумывает случай (может быть, сверяется с правилами), но в итоге сообщает игроку, что решение окончательное.
	\item При затруднении принятия решения, судья зовет на помощь главного судью. Главный судья выносит окончательное решение. Все возражения принимаются после игры. 
\end{itemize}

Не допускается давление на судью со стороны игроков ни в каком виде. Игрок не является авторитетом при разборе спорных ситуаций, поэтому решения судьи должны выполняться неукоснительно.

\subsection{Играющие судьи}

Допускается, что судьи могут принимать участие в турнире. На больших (80 и более человек) турнирах обязательно наличие судей, которые не принимают участие в турнире из расчета одного судьи на 10 столов, возле которых нет играющих судей.

Играющий судья может обслуживать максимум 5 соседних столов. В случае возникновения спорной ситуации за столом, где играет судья, стол обязан позвать другого судью для исключения предвзятости при вынесении решения. 

В случае, если играющий судья задерживается на вынесении решения, стол, за которым играет судья, должен получить дополнительные минуты игры. Такая ситуация в среднем рассматривается как форс-мажор (т.е. не должна происходить часто).

\subsection{Помехи при игре}

Игроки обязаны действовать разумным образом в течение всей игры, чтобы не допускать появления спорных ситуаций.

Примеры действий игроков, которые могут привести к спорным ситуациям:
\begin{itemize}
	\item Выкладывание тайлов в неположенные места (цумо близко к дискарду; объявленные сеты слева от руки)
	\item Расположение тайлов руки неположенным образом (например, закрытие руки)
	\item Слишком быстрая игра. Необходимо немного подождать перед взятием тайла со стены, чтобы все игроки могли увидеть только что сброшенный тайл и сделать соответствующие объявления.
	\item Слишком медленная игра. В случае, если игрок регулярно затягивает свои сбросы, это может трактоваться как умышленная помеха для игры. Игроки вправе вызвать судью для вынесения вежливого указания играть быстрее.
	\begin{itemize}
		\item В случае, если игроку требуется больше времени для принятия решения об объявлении, игрок вправе попросить других игроков немного подождать (но не более 10 секунд).
	\end{itemize}
	\item Шум и провокационное поведение при игре. Несмотря на то, что мы не вправе полностью запретить любое общение за столом в текущем регламенте, мы ожидаем, что игроки не будут мешать соседним столам своими разговорами. Также мы ожидаем, что игроки не будут никоим образом подначивать оппонентов на совершение тех или иных действий, вести себя вызывающе и каким-либо иным образом пытаться влиять на ход игры посредством коммуникаций.
\end{itemize}

Телефоны во время игры должны быть переведены в беззвучный режим, чтобы не мешать игрокам. Не допускается ни в какой форме голосовое общение и использование текстовых сообщений --- в случае нарушения судья может назначить штраф от 8000 до 12000 очков после умы, в случае повторного нарушения --- на усмотрение судьи. В случае, если игрок ожидает важный звонок, он должен предупредить об этом судейский состав и игроков за столом. На время звонка игроку следует выйти из зала, чтобы не мешать другим своим разговором.

Использование наушников во время игры не запрещается, при условии что игрок ясно слышит все объявления за столом. В случае возникновения спорной ситуации из-за того, что игрок не услышал объявление, игрок обязан убрать наушники до конца игры.

Игроки должны позаботиться о том, чтобы других не отвлекали в том числе посторонние запахи:
\begin{itemize}
	\item Использовать леденцы или жвачку после курения в перерыве. За столом могут быть игроки, не переносящие запах табака, уважайте их право.
	\item Любые резкие запахи, исходящие от игрока, являются веской причиной для отстранения игрока от турнира, если они не могут быть быстро устранены.
\end{itemize}

Появление игрока на турнире в состоянии алкогольного или иного опьянения недопустимо. Игроки в состоянии опьянения должны быть немедленно дисквалифицированы.

\subsubsection{Использование иных вещей игрока во время игры}

Игрок вправе принести за стол плотно закрывающуюся емкость с питьевой водой или иным безалкогольным напитком. Не допускается приносить за стол любую еду, напитки в стаканах и иных неплотно закрывающихся емкостях. В случае нарушения, судья или организатор должен немедленно унести все из вышеописанного и вежливо попросить больше так не делать. При повторном нарушении судья вправе назначить произвольный штраф.

Не допускается наличие на столе никаких вещей игроков, не относящихся к игре. Сумки следует ставить на пол рядом со стулом таким образом, чтобы они не мешали проходу судей между столами. Верхнюю одежду следует оставить в гардеробе, но также допустимо повесить на стул.

Допустимо использовать мобильный телефон в случае проведения турнира с использованием электронных средств подсчета очков. Если турнир играется на палочках, использование мобильного телефона во время игры не допускается ни в каком виде.

\subsubsection{Выход из-за стола во время игры}

Игрокам следует по возможности избегать необходимости выходить из-за стола в процессе игры. Если такая необходимость все же возникла, игроку следует вежливо попросить оппонентов подождать. В случае неоднократного повторения к такому игроку могут применены санкции на усмотрение судьи.

Оптимальным вариантом для выхода из-за стола считается время между раздачами. Игроку следует попросить других игроков замешать тайлы и построить стены во время его отсутствия с целью уменьшения задержки насколько это возможно.

Не следует выходить из-за стола без существенной причины.

\subsubsection{Передача информации во время игры}

Передача информации тем или иным образом карается штрафом на усмотрение судьи (независимо от того, насколько эта информация правдива). Примеры нарушений:
\begin{itemize}
	\item Обсуждение собственной руки вслух;
	\item Обсуждение рук и дискардов других игроков, попытки анализа дискардов вслух.
\end{itemize}

Не приветствуются, но и не наказываются попытки передачи информации, которая очевидна для всех игроков, например:
\begin{itemize}
	\item Указание на количество тайлов в руке;
	\item Показное или неконтролируемое волнение при сбросе опасного тайла.
\end{itemize}

Также не наказываются манипуляции с собственной рукой:
\begin{itemize}
	\item Карагири (сброс такого же тайла, что был взят, но из другой части руки);
	\item Умышленное отсутствие сортировки в руке;
	\begin{itemize}
		\item Заметим, что игрок, не сортирующий руку в процессе раздачи, все еще обязан ее отсортировать при победе, чтобы яку и ожидания в руке были очевидны для всех игроков;
	\end{itemize}
	\item Привычка переставлять тайлы произвольным образом, играться с тайлами также не наказывается, однако оппоненты могут попросить так не делать, если это отвлекает от игры или нервирует.
\end{itemize}

Существуют случаи передачи информации, которые не только допускаются, но и приветствуются в соответствии с духом спортивного соревнования, например:
\begin{itemize}
	\item Указание игроку на то, что он взял тайл не с той стороны стены;
	\item Указание игроку на то, что в его руке неправильное количество тайлов, что ведет к мертвой руке;
	\item Указание игроку на очевидно неправильные действия (например, вскрытие тайлов из руки, не составляющих корректный сет при объявлении, в этом случае допускается исправить сет);
	\item Озвучивание чужих и собственных сбросов в начале раздачи и в случае, если кто-либо за столом отвлекся и мог не увидеть сброс;
	\begin{itemize}
		\item Если игрок озвучивает сброс некорректно, это может быть поводом для вежливого указания так не делать;
	\end{itemize}
	\item Озвучивание открытых индикаторов дор.
\end{itemize}

\subsection{Штрафные санкции}

Предусматриваются следующие виды санкций в отношении игрока:
\begin{itemize}
	\item Вежливые указания (отсутствие штрафа):
	\begin{itemize}
		\item Применяются в случае, если требуется указать игроку на то, чтобы он сделал что-либо правильным образом (либо перестал что-то делать неправильным образом);
		\item Могут вести к ужесточению наказания в случае, если игрок не прислушался к вежливому указанию (см. ниже);
	\end{itemize}
	\item Мертвая рука:
	\begin{itemize}
		\item Применяется в случае, если имело место серьезное нарушение, но при этом игра все еще может продолжаться;
		\item Игрок с мертвой рукой не имеет права делать никаких объявлений (под угрозой ужесточения наказания);
	\end{itemize}
	\item Чомбо:
	\begin{itemize}
		\item Применяется в случае, если серьезное нарушение привело к тому, что раздачу продолжать невозможно;
		\item Размер штрафа чомбо --- 20000 очков после умы;
	\end{itemize}
	\item Дисциплинарный штраф:
	\begin{itemize}
		\item Применяется в случае, если игрок регулярно игнорирует вежливые указания судьи;
		\item Размер штрафа оставляется на усмотрение судьи;
		\item Штраф вычитается из итоговых очков игрока после умы;
	\end{itemize}
	\item Штрафы за опоздание:
	\begin{itemize}
		\item Применяется в случае опоздания игрока на очередную игру;
		\item Рассчитывается как 1000 очков за каждую минуту опоздания;
		\item После 10 минут опоздания вместо игрока садится игрок замены. Если игрок явился к столу позднее чем через 10 минут, он вправе продолжить игру вместо игрока замены, но при расчете итогового рейтинга после этой игры его результат будет учтен так же, как был бы учтен результат игрока замены;
	\end{itemize}
	\item Штрафы за умышленные препятствующие действия:
	\begin{itemize}
		\item Могут составлять 8000 или 12000 очков после умы в зависимости от тяжести нарушения;
	\end{itemize}
	\item Злостные и повторяющиеся случаи нарушений:
	\begin{itemize}
		\item Штраф может составлять от 12000 до 48000 очков после умы в зависимости от тяжести нарушения;
	\end{itemize}
	\item Дисквалификация игрока с турнира:
	\begin{itemize}
		\item Применяется в случае особо злостных нарушений;
		\item Также может применяться в случае неявки игрока на две игры подряд;
		\item Решение о дисквалификации игрока не только с текущего, но и с последующих турниров принимается судейской комиссией в зависимости от обстоятельств.
	\end{itemize}
\end{itemize}

\subsubsection{Порядок увеличения штрафов}

При вынесении вежливого указания судья обязан объяснить игроку, что он делает не так и как нужно поступать правильно.

В случае, если уже было вынесено вежливое указание, но игрок повторяет действия, приводящие к спорным ситуациям, судья вправе назначить игроку произвольный штраф. Штрафы увеличиваются последовательно:

\begin{itemize}
	\item За первое повторение после указания --- 2000 очков
	\item За второе повторение --- 4000 очков
	\item За третье повторение --- 8000 очков
	\item За четвертое и последующие повторения --- 16000 очков с увеличением с два раза за каждый следующий повтор.
\end{itemize}

Все штрафы и причины штрафов должны фиксироваться в судейском журнале. За повторное нарушение по каждой причине штраф увеличивается отдельно.

\subsubsection{Случаи, допускающие самостоятельное решение спорной ситуации}

Обычно на турнире требуется звать судью для фиксации любого нарушения. Однако поскольку судей может быть ограниченное количество, некоторые ситуации могут быть разрешены игроками самостоятельно, такие ситуации обязаны отвечать следующим требованиям:
\begin{itemize}
	\item Ситуация однозначно приводит к штрафным санкциям в виде мертвой руки или чомбо;
	\item Все игроки за столом, включая игрока, попадающего под штрафные санкции, могут легко проверить, что имеет место нарушение, и согласны с тем, какие именно штрафные санкции следует применить.
\end{itemize}

Примеры разрешимых ситуаций:
\begin{itemize}
	\item У одного из игроков на руке оказалось неверное количество тайлов. Эта ситуация ведет к мертвой руке и легко проверяема всеми игроками за столом.
	\item Игрок объявил фуритен рон и открыл руку. Эта ситуация ведет к чомбо и также легко проверяема всеми игроками за столом.
\end{itemize}

Игроки обязаны позвать судью при любом из следующих условий:
\begin{itemize}
	\item У игроков есть сомнения относительно того, какие именно санкции должны быть применены;
	\item Среди игроков нет согласия относительно того, какие санкции должны быть применены;
	\item Кто-либо из игроков настаивает на применении штрафных санкций, но остальные не уверены в корректности этого требования;
	\item Нарушение несущественное и не приводит к штрафным санкциям, но мешает нормальной игре;
	\item Произошло злостное нарушение.
\end{itemize}

\subsubsection{Произвольное вмешательство судьи}

Если судья замечает явное нарушение правил или требований за каким-либо столом, он имеет право вмешаться сразу же, не дожидаясь возникновения спорной ситуации.

Если судья замечает нарушение рекомендаций, он не должен вмешиваться до тех пор, пока игроки не обратят на это внимание судьи.

\subsection{Подсчет очков}

Игроки обязаны считать свои очки самостоятельно и по максимальной стоимости. Стол может (но не обязан) поправить игрока в случае, если он неправильно посчитал свои очки. Перерасчеты после проведения выплат не допускаются.

Если игроки за столом не уверены в том, как именно рассчитать очки, они вправе позвать судью на помощь. Впрочем, такое должно происходить редко.

\subsection{Проблемное поведение игроков}

Судье не следует поддаваться ни на какие провокации со стороны игроков. Недопустимо вступать с игроками в споры и дискуссии. 

Проблемными следует считать игроков:
\begin{itemize}
	\item Игнорирующих вежливые указания судьи;
	\item Регулярно вступающих в споры относительно судейских решений;
	\item Пытающихся давить на судью или запугивать его;
	\item Регулярно зовущих судью по мелочам (особенно по таким, которые игроки могут разрешить самостоятельно);
	\item Апеллирующих к нюансам правил с целью получения выгоды (в частности, в случае просьбы игрока назначить наказание, не соответствующее нарушению).
\end{itemize}

Судейский состав должен обмениваться сведениями о том, кто из игроков является потенциально проблемным. Идеальный вариант --- ведение судейского журнала с фиксацией всех указаний и нарушений.

Наказания за нарушения для проблемных игроков следует назначать в среднем более строгие, чем всем остальным (в случаях, когда наказание оставляется на усмотрение судьи).

\subsection{Игроки замены и пропуск игр}

В случае неявки игрока к началу игры игроку начисляется штраф в 1000 очков за каждую минуту опоздания. В случае, если игрок не явился через 5 минут после начала игры, вместо игрока садится игрок замены, и за эту игру игрок получает фиксированный штраф независимо от занятого места по итогу игры.

Также следует посадить игрока замены и начислять ему фиксированный штраф за каждую игру в следующих случаях:
\begin{itemize}
	\item Если игрок вынужден покинуть турнир из-за обстоятельств непреодолимой силы;
	\item Если игрок был дисквалифицирован с турнира за неспортивное поведение.
\end{itemize}

Размер фиксированного штрафа вычисляется в зависимости от умы, принятой на турнире, и в общем случае должен быть равен удвоенному штрафу умы за последнее место в игре. Например, если ума 15/5, то фиксированный штраф должен составлять 30000 очков, если же ума 30/10, то штраф будет уже 60000 очков. Таким образом обеспечивается более справедливое распределение очков.

В случае, если замена происходит в середине раздачи, раздачу следует прервать и переиграть заново.

В случае, если игрок пропускает половину или более игр турнира, судья имеет право исключить игрока из турнира полностью и не учитывать его результаты в общем рейтинге. Решение о санкциях в отношении игрока, досрочно покинувшего турнир, остается за судейской коллегией, в частности, игрок, покинувший турнир по неуважительной причине, может быть подвергнут дисквалификации на последующие турниры в течение определенного периода времени.

\subsubsection{Сокращение количества столов}

В случае, если замена игрока перед игрой приводит к тому, что на турнире оказывается 4 игрока замены, организаторы турнира могут решить сократить количество игровых столов, чтобы освободить игроков замены на случай иных непредвиденных ситуаций.

Однако следует иметь в виду, что недопустимо дальнейшее увеличение количества столов (например, если игрок решил вернуться на турнир, а количество столов уже было сокращено). Следует предупредить игрока о том, что его уход с турнира повлечет за собой его выбывание.

В общем случае, считается недопустимым увеличивать количество столов из-за того, что игроки решили прийти на турнир после его начала.

\subsection{Порядок вынесения решений}

Судья должен придерживаться следующего порядка выяснения обстоятельств и вынесения решения:
\begin{itemize}
	\item Посмотреть на ситуацию на столе и по возможности понять, что произошло;
	\item Поговорить со всеми игроками за столом и выслушать их аргументы; следует не допускать давления кого-либо из игроков на судью;
	\item Вынести решение и объяснить игрокам, почему это решение было вынесено. Важно, чтобы все игроки за столом понимали происходящее и никто не чувствовал, что его не выслушали.
	\item Внести запись о нарушении, нарушителе и вынесенном решении в судейский протокол. Протокол может быть как обычным бумажным списком, так и электронным. Необходимость протоколирования любых решений требуется для последующей ретроспективной оценки работы судьи, а также для фиксации повторяющихся нарушений и как следствие --- назначения более строгих наказаний.
\end{itemize}

\subsection{Обоснования для присуждения штрафов}

Основной принцип при присуждении штрафа --- чем большую помеху представляет нарушение для игрового процесса, тем строже наказание.

Размер штрафа не должен зависеть от исходных условий, например:
\begin{itemize}
	\item Если игрок нервничает или просто неловок;
	\item Если игрок неопытен и нарушает правила по незнанию;
\end{itemize}

Если игрок нарушает правила умышленно, наказание всегда должно быть строгим, однако наличие умысла должно быть однозначно доказуемым.

В следующей таблице приведены категории нарушений и соответствующие наказания. По умолчанию предполагается использовать столбец "По регламенту", однако при наличии смягчающих обстоятельств судья вправе назначить более мягкое наказание. Выяснение всех обстоятельств нарушения и принятие решения остается за судьей. 

Примечание: если в колонке "Минимальные санкции" стоит прочерк, это означает, что смягчение наказания не предусматривается.

\noindent\begin{tabularx}{\linewidth}{L{0.3\linewidth}L{0.3\linewidth}L{0.3\linewidth}}
	\caption{Обоснования штрафов} \\
	\toprule
	\textbf{Нарушение} & \textbf{По регламенту} & \textbf{Минимальные санкции} \\
	\endfirsthead
	\toprule
	\textbf{Нарушение} & \textbf{Согласно правилам} & \textbf{Минимальные санкции} \\
	\midrule
	\endhead
	\multicolumn{3}{r}{\footnotesize(Продолжение на следующей странице)}
	\endfoot
	\bottomrule
	\endlastfoot

	\multicolumn{3}{c}{\cellcolor{gray!25}Перемешивание и взятие тайлов} \\
	Ошибки при раздаче тайлов &
	Перераздача &
	- \\
	\midrule
	Недостаточное или избыточное количество тайлов на руке &
	Мертвая рука &
	- \\
	\midrule
	Некорректное взятие тайла &
	Мертвая рука &
	Вежливое указание \\
	\midrule
	Некорректный показ тайлов & 
	Мертвая рука &
	Вежливое указание \\
	\midrule
	Взятие тайла не в свой ход & 
	Мертвая рука &
	Вежливое указание и возврат тайла на место (если возможно установить какой именно тайл был взят) \\
	\midrule
	Взятие тайла из некорректного места стены & 
	Мертвая рука &
	Вежливое указание и возврат тайла на место (если возможно установить какой именно тайл был взят) \\
	\midrule
	Взятие выигрышного тайла из дискарда соперника &
	Вежливое указание &
	- \\
	\midrule
	Взятие тайла из руки соперника &
	Чомбо &
	Если соперник закрыл руку (чего делать нельзя) и из нее после этого взяли тайл, спутав со стеной, санкции остаются на усмотрение судьи. \\
	\multicolumn{3}{c}{\cellcolor{gray!25}Объявления} \\
	Ошибочное объявление любого сета без вскрытия тайлов либо отмена объявления сета &
	Вежливое указание &
	- \\
	\midrule
	Объявление любого сета, ошибочность объявления замечена сразу после открытия\footnote{Если игрок вскрывает тайлы, которые не образуют сет с последним тайлом в дискардах, либо вскрывает четыре тайла, которые не образуют кан} &
	Вежливое указание и отмена объявления. Допускается также скорректировать объявление, открыв корректный тайл и закрыв ошибочный, в этом случае объявление не отменяется. &
	- \\
	\midrule
	Отмена объявления сета после вскрытия тайлов, если объявление технически корректно &
	Вежливое указание. Отмена объявления считается недействительной. &
	- \\
	\midrule
	Ошибочное объявление риичи &
	Мертвая рука &
	Вежливое указание \\
	\midrule
	Ошибочное объявление победы без открытия руки &
	Мертвая рука &
	- \\
	\midrule
	Ошибочное объявление победы с открытием руки &
	Чомбо & 
	- \\
	\midrule
	Любое объявление при мертвой руке & 
	Чомбо & 
	- \\
	\midrule
	Любое объявление при мертвой руке, если мертвая рука еще не объявлена &
	Мертвая рука &
	- \\
	\multicolumn{3}{c}{\cellcolor{gray!25}Открытые сеты} \\
	Невалидный открытый сет в руке &
	Мертвая рука &
	- \\
	\midrule
	Смещение сета / нарушение куикаэ &
	Мертвая рука &
	- \\
	\midrule
	Некорректное размещение тайлов в открытых сетах или в дискарде &
	Вежливое указание &
	- \\
	\multicolumn{3}{c}{\cellcolor{gray!25}Риичи} \\
	Неповернутый тайл сброса при объявлении &
	Вежливое указание &
	- \\
	\midrule
	Нотен риичи &
	Чомбо &
	- \\
	\midrule
	Открытие ожидания или всей руки при риичи &
	Мертвая рука &
	- \\
	\midrule
	Замена тайла в руке при риичи &
	Мертвая рука. В случае если игра доходит до ничьей, следует назначить чомбо независимо от наличия темпая в руке. Если есть возможность проконтролировать, что был сброшен именно тот тайл, который был взят со стены, чомбо в конце раздачи не применяется. &
	- \\
	\midrule
	Объявление закрытого кана, меняющее интерпретацию структуры руки &
	Чомбо & 
	- \\
	\multicolumn{3}{c}{\cellcolor{gray!25}Неспортивное поведение} \\
	Поведение, мешающее раздаче &
	На усмотрение судьи &
	- \\
	\midrule
	Посторонние объекты на столе (не предусмотренные регламентом) &
	На усмотрение судьи & 
	- \\
	\midrule
	Передача информации (в т.ч. разглашение информации о собственной руке) &
	На усмотрение судьи &
	- \\
	\midrule
	Жульничество &
	Немедленная дисквалификация &
	- \\
	\midrule
	Подсматривание тайлов в чужой руке &
	Чомбо &
	- \\
	\multicolumn{3}{c}{\cellcolor{gray!25}Опоздания} \\
	Опоздание не более чем на 10 минут к началу игры &
	Штраф 1000 очков после умы за каждую минуту опоздания &
	- \\
	\midrule
	Опоздание более чем на 10 минут к началу игры &
	Замена игрока &
	Игрок может сесть доигрывать игру вместо замены, но в конце игры все еще получит штраф, как если бы всю игру играл игрок замены \\
	\midrule
	Пропуск ханчана &
	Замена игрока &
	- \\
	\midrule
	% TODO: check page breaks
	\linebreak\linebreak\linebreak\linebreak\linebreak & & \\
	\multicolumn{3}{c}{\cellcolor{gray!25}Вскрытие тайлов} \\
	Вскрытие 1-3 тайлов из мертвой стены &
	Мертвая рука &
	В случае случайного вскрытия единственного тайла может быть применено вежливое указание. Тайл должен быть возвращен на место. \\
	\midrule
	Вскрытие более 3 тайлов мертвой стены &
	Чомбо &
	- \\
	\midrule
	Вскрытие тайлов из живой стены &
	Чомбо &
	В случае если вскрыто незначительное количество тайлов, судья может решить, что раздача может продолжаться как есть. \\
	\midrule
	Подсматривание урадор до конца раздачи &
	Чомбо &
	- \\
	\midrule
	Показ незначительного количества тайлов из собственной руки до конца раздачи &
	Вежливое указание &
	- \\
	\midrule
	Показ значительного количества тайлов из собственной руки или всей руки до конца раздачи. Показ более половины закрытых тайлов в руке считается значительным. &
	Мертвая рука &
	- \\
	\midrule
	Вскрытие любого числа тайлов из руки соперника &
	Чомбо &
	Если соперник не объявил риичи, вскрыто не более 2 тайлов по неосторожности И если в закрытой части руки соперника более 4 тайлов, можно ограничиться вежливым указанием. \\
	\midrule
	Вскрытие некорректного индикатора доры &
	Вежливое указание; вскрытие корректного индикатора.
	
	В случае если игроки еще не видели своих рук, стол вправе устроить перераздачу. &
	- \\
	\midrule
	% TODO: check page breaks
    & & \\
	\multicolumn{3}{c}{\cellcolor{gray!25}Прочее} \\
	Закрытие руки &
	Вежливое указание &
	- \\
	\midrule
	Стол играет слишком шумно &
	На усмотрение судьи &
	Вежливое указание \\
	\midrule
	Вскрытие тайлов в стенах после объявления победы &
	Вежливое указание &
	- \\
	\midrule
	Неаккуратный сброс, в том числе хаос в дискарде (тайлы, лежащие на расстоянии друг от друга, тайлы, лежащие вне зоны дискарда) &
	Чомбо (в случае если из-за неаккуратного сброса произошло необратимое изменение порядка дискарда или смешивание дискардов, а также в случае существенного раскрытия тайлов из одной или нескольких стен) &
	Вежливое указание \\
	\midrule
	Замешивание тайлов до того, как произошел взаиморасчет &
	См. примечание ниже &
	- \\
\end{tabularx}

В случае начала замешивания тайлов до взаиморасчета, следует определить, возможно ли провести взаиморасчет на основе текущего состояния стола. Если это невозможно, игроку, замешавшему тайлы, назначается чомбо, в ином случае следует учесть столько информации, сколько возможно. В случае, если из-за замешивания дискардов не удается определить, имел ли место временный или постоянный фуритен для выигравшего игрока, замешавшему игроку также назначается чомбо.

Игроку, замешавшему тайлы мертвой стены, следует выписать штраф за неспортивное поведение. Размер штрафа оставляется на усмотрение судьи, но составляет не менее 8000 очков после умы. В случае замешивания мертвой стены до того как игроки увидели урадоры (при победе по риичи), судья имеет право решить вытянуть из замешанной части столько тайлов, сколько было урадор и учесть их в результате. 

В иных случаях, не препятствующих проведению взаиморасчета, замешивание рук и дискардов игроков, не участвующих во взаиморасчете, также не приветствуется, но на первый раз наказания не следует. Судья должен вынести вежливое указание не делать так в дальнейшем.

\subsection{Стили судейства}

Наиболее распространенные стили судейства:
\begin{itemize}
	\item "Вездесущий" судья --- следит за всеми сразу, много вмешивается, если видит нарушения;
	\item "Невидимый" судья --- приходит на помощь только в случае затруднений и нарушений.
\end{itemize}

Ожидается, что игроки сами в состоянии разрешить распространенные спорные ситуации и не будут звать судью по каждой мелочи. Таким образом, в данном регламенте мы склоняемся к невидимому стилю судейства.

Можно выделить несколько типичных ситуаций, в которых судье необходимо подойти к столу и разобраться в ситуации:
\begin{itemize}
	\item Игроки явным образом зовут судью;
	\item Игроки заметно спорят;
	\item За столом заметно существенное замешательство;
	\item За столом наблюдается препятствование игре или жульничество.
\end{itemize}

Независимо от стиля судейства, в случае если судья заметил явное нарушение, он обязан вмешаться.

Судья должен уделять внимание всем столам в своей зоне ответственности, однако он также может решить уделять больше внимания некоторым столам --- например, столам с проблемными игроками или столам, за которыми играют неопытные игроки.

Судье следует заботиться о том, чтобы не давать намеков о составе рук при выполнении судейских обязанностей.

\subsection{Получение и отзыв статуса судьи}

Судьи турниров обязуются получить аккредитацию на выполнение судейских обязанностей на турнирах. Аккредитация проводится любым из уже аккредитованных судей. В процессе аккредитации судья делает презентацию по данному разделу, а также проводит тестирование претендентов.

Тестирование проводится в форме ряда вопросов с открытыми ответами. По результатам тестирования в получении статуса судьи может быть отказано в случае, если некорректные ответы даны на более чем треть вопросов. Список действующих судей приводится на сайте сообщества риичи в РФ.

Судейский семинар может проводиться как в формате очной встречи, так и онлайн. В случае очной встречи, презентация проводится очно и сразу после презентации следует тестирование. В случае онлайн-аккредитации, претенденту предлагается просмотреть запись презентации и пройти онлайн-тестирование с дальнейшей беседой по удаленной связи. 

В случае, если к судье возникает слишком много вопросов, статус судьи может быть отозван коллегиальным решением большинства всех действующих судей. Порядок рассмотрения запросов на отзыв статуса судьи:

\begin{itemize}
	\item Получение массовых жалоб на конкретного судью. Порядок подачи жалоб и апелляций рассмотрен ниже.
	\item Вынесение жалоб на рассмотрение судейской комиссии. Комиссия выносит решение о том, насколько критичны и насколько обоснованы поступившие жалобы. На данном этапе процесс может быть прекращен в случае недостаточной обоснованности или критичности жалоб.
	\item Оповещение судьи о том, какие действия ему следует предпринять, чтобы исправить проблемы, на которые указали игроки. Кроме этого, следует назначить дополнительного судью на все грядущие турниры для контроля выполнения пожеланий.
	\item В случае, если судья не прислушался к замечаниям, судья-наблюдатель ставит на обсуждение вопрос об отзыве судейской аккредитации.
	\item Судейская коллегия рассматривает отзыв судьи-наблюдателя и принимает решение об отзыве судейской аккредитации.
\end{itemize}

\subsection{Апелляции и жалобы}

В случае, если игрок не согласен с решением судьи, либо к поведению судьи есть претензии, игрок вправе оставить жалобу/апелляцию. Апелляции и жалобы принимаются строго после окончания турнира.

В тексте апелляции/жалобы следует максимально подробно описать спорную ситуацию, вынесенное судьей решение и причины, по которым игрок не согласен с вынесенным решением. Апелляции/жалобы с недостаточно подробным описанием к рассмотрению не принимаются.

По итогу рассмотренной апелляции игрок имеет право на изменение итоговых рейтинговых очков в турнире, если судейская коллегия решит, что вынесенное решение не соответствует регламенту (например, если судья выписал штраф чомбо вместо назначения мертвой руки). Размер компенсации определяется судейской коллегией. 

Не принимаются апелляции и жалобы, касающиеся турниров, для которых не применялся данный регламент. Кроме того, устанавливается срок давности в два месяца для рассмотрения жалоб и апелляций, таким образом жалобы по турнирам, проведенным более чем два месяца назад, также не принимаются.

Рассмотрение жалоб и апелляций может занимать некоторое время, однако рекомендуется рассмотреть все жалобы в течение месяца после подачи. 

После рассмотрения жалобы или апелляции, судейская коллегия обязуется сообщить игроку, подавшему жалобу или апелляцию, о принятом решении.