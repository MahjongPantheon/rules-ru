\section{Особенности проведения онлайн-турниров}
	
Турниры проводятся только на поддерживаемых онлайн-платформах (Mahjong soul, Tenhou net). 

\subsection{Общие положения}

Организаторы онлайн-турниров берут на себя обязательство по приведению правил турнира в соответствие с текущим регламентом. Если по какой-то причине не удается установить полное соответствие правил, либо если предполагаются отклонения от правил, организаторы обязаны сообщить об этом в сопутствующем регламенте онлайн-турнира.

В случае, если на турнир зарегистрировано некратное четырем число игроков, организаторы добавляют игроков замены в виде игровых ботов. Техническое обеспечение ложится на организаторов турнира.

Формирование рассадки для очередного тура является обязанностью организаторов турнира. Организаторы могут использовать любые технические средства автоматизации для формирования рассадок.

Игрок обязан участвовать в турнире с единственного игрового аккаунта. Каждый игровой аккаунт должен быть однозначно сопоставлен с конкретным игроком для последующего внесения в онлайн-рейтинг.

Координация турнира происходит через общий организационный чат, в который должны быть добавлены все участвующие игроки. В чате также могут присутствовать наблюдатели.

При общении в организационном чате игроки обязаны придерживаться базовых правил приличия (исключить оскорбления, переходы на личности, разжигание ненависти по любому признаку, флуд).

\subsection{Нарушения правил, специфические для онлайн-турниров}

\begin{itemize}
	\item Игрок, замеченный в одновременном использовании более чем одного аккаунта при игре в турнире, немедленно дисквалифицируется с турнира. При повторном нарушении на последующих турнирах, к игроку могут быть применены санкции вплоть до полного запрета на участие в онлайн-турнирах.
	\item Игроки, замеченные в сговоре, немедленно дисквалифицируются с турнира. При повторном нарушении на последующих турнирах, к игрокам могут быть применены санкции вплоть до полного запрета на участие в онлайн-турнирах.
	\item В случае, если турнир транслируется на стриминговом сервисе кем-либо из игроков, игрок обязан поставить задержку стрима не менее чем в 5 минут во избежание подглядывания за рукой игрока соперниками. 
	\begin{itemize}
		\item В случае, если это правило нарушено, никаких претензий относительно подглядывания другими игроками не принимается.
		\item В отдельных случаях отсутствие задержки на стриме может трактоваться как сговор (см. предыдущий пункт) со всеми соответствующими последствиями.
	\end{itemize}
	\item Организатор вправе дисквалифицировать игрока за откровенно неадекватное поведение в игре или в общем чате. В особо тяжелых случаях игрок также может быть заранее дисквалифицирован с последующих турниров. Срок дисквалификации определяется организатором турнира.
\end{itemize}

\subsection{Форс-мажоры}

В случае, если кто-либо из игроков самовольно покидает турнир до его завершения, к игроку могут быть применены санкции:

\begin{itemize}
	\item При наличии уважительной причины, вместо игрока добавляется игрок замены, получающий фиксированный штраф за каждую пропущенную игру.
	\item Организаторы имеют право сократить количество столов в случае, если суммарное количество игроков замены в турнире равно или превышает 4. В этом случае для всех игроков, которые не играют в очередном туре, применяется штраф за пропуск игры.
	\item В случае отсутствия уважительной причины, организаторы вправе дисквалифицировать игрока на определенное количество последующих онлайн-турниров, проводимых этими организаторами. Конкретное количество турниров определяется организаторами.
\end{itemize}

В случае технических неполадок на платформе, организаторы вправе досрочно завершить турнир, либо перенести оставшиеся игры на другое время. При досрочном завершении турнира, его результаты не учитываются в общем онлайн-рейтинге, турнир считается непроведенным.

Если из-за временных технических неполадок на стороне игрока он не находится в игровом лобби на момент старта его игры, игроку дается 5 минут на решение проблемы. Если неполадки не удается устранить за 5 минут, игрок заменяется на бота на время текущей игры. Рекомендуется заходить в игровое лобби заранее, чтобы было время на решение проблем в случае их возникновения.

\subsection{Технические регламенты турниров}

Техническое обеспечение разных онлайн-турниров может существенно различаться и поэтому не регламентируется в данном своде правил. Под техническим обеспечением здесь и далее подразумеваются:

\begin{itemize}
	\item Платформа для проведения игр;
	\item Организационный чат;
	\item (опционально) Средства автоматизации турнира, позволяющие провести рассадку и начать игровую сессию, например:
	\begin{itemize}
		\item Встроенные в игровую платформу средства;
		\item Бот-помощник в организационном чате;
	\end{itemize}
	\item (опционально) Рейтинговая система, например:
	\begin{itemize}
		\item Pantheon --- для полностью автоматического учета результатов;
		\item Google spreadsheets --- для ручного учета результатов;
	\end{itemize}
	\item Иные технические средства автоматизации турниров, в случае их наличия.
\end{itemize}


Организаторы турнира обязуются заранее выложить сопутствующий технический регламент, зависящий от того, какое техническое обеспечение используется на турнире. Регламент должен включать пояснения по следующим моментам:
\begin{itemize}
	\item Как корректно зарегистрироваться на турнир;
	\item Когда и где играется турнир (время с указанием часового пояса + ссылка для входа в лобби);
	\item Нужны ли дополнительные действия от игроков для учета их результатов в турнирном рейтинге, и если нужны то какие именно;
	\item Описание возможных форс-мажоров для выбранного технического обеспечения, способы их решения.
\end{itemize}
