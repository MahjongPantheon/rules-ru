\section{Использование электронных ассистентов}

Использование электронных ассистентов в клубных играх правилами \textbf{не регламентируется}. Далее рассмотрены правила и рекомендации по использованию ассистентов в турнирных играх.

\subsection{Общие требования к турниру и к игрокам}

У каждого игрока, имеющего мобильное устройство для использования ассистента, должен быть стабильный доступ в интернет с этого устройства.

Организаторы турнира должны позаботиться о наличии стабильного WiFi-соединения с выходом в интернет. Это особенно важно в случае присутствия на турнире иностранных игроков, у которых может просто не быть местной сим-карты с доступом в интернет. Если иностранных игроков на турнире нет, наличие WiFi оставляется в качестве рекомендации.

Организаторский состав должен иметь как минимум одно общее устройство (ноутбук или компьютер) для администрирования турнира. Также с этого устройства может выводиться таймер и рассадка на проектор или экраны в зале, в случае их наличия.

Игрокам следует \textbf{заблаговременно осуществить вход в систему}. До начала турнирного дня следует проверить, что ассистент готов к использованию --- проверить, что вход осуществлен и что в настройках выбран текущий турнир.

\textbf{Не рекомендуется} использовать инкогнито-режим браузера для входа в ассистент, поскольку при случайном закрытии вкладки потребуется заново входить в систему и выбирать турнир в настройках.

Игрок вправе положить свой телефон в центр стола на время раздачи. Если кто-либо из игроков за столом против того, чтобы телефон находился в центре, телефон следует убрать. На время перемешивания тайлов телефон также следует убрать со стола.

\subsection{Правила проведения турнира при использовании ассистента}

В общем случае использование мобильных телефонов и иных средств связи на турнирах запрещено правилами.

В случае, если на турнире предполагается использование программных средств для учета раздач и построения рейтингов, требуется придерживаться следующих правил:

\begin{itemize}
	\item Мобильным устройством допускается пользоваться \textbf{исключительно} для следующих действий:
	\begin{itemize}
		\item Внесение раздачи и предпросмотр результатов;
		\item Просмотр очков игроков за столом;
		\item Просмотр лога игры (предыдущих внесенных раздач) для контроля правильности внесенных данных.
	\end{itemize}
	\item В случае использования мобильного устройства для действий помимо описанных выше, игроки должны попросить так не делать, в случае повторения инцидента следует позвать судью для получения вежливого указания и/или штрафа.
	\item При внесении раздачи игрок проговаривает вслух: были ли объявлены риичи и кем, кто выиграл, с кого в случае рона, открытая ли рука, все яку, доры, фу, чтобы все могли убедиться что ничего не было забыто. Аналогичные действия нужно делать и при ничьей. После внесения результата игрок \textbf{обязан} показать экран предпросмотра всему столу, чтобы игроки могли увидеть перераспределение очков и в случае необходимости попросить внести коррективы.
	\item Допускается положить телефон в центр стола, чтобы игроки могли видеть свои очки и таймер в процессе раздачи.
	\item Не у всех игроков может быть подходящее мобильное устройство. В случае, если игроки просят кого-либо с мобильным устройством показать очки, игрок \textbf{не вправе} отказать.
	\item В случае, если было обнаружено, что предыдущая раздача внесена некорректно, игроки должны \textbf{позвать судью} и попросить отменить раздачу.
	\item Если вносящий игрок не умеет пользоваться ассистентом, игроки за столом обязаны помочь ему с внесением раздач и, в идеале, обучить его правильному внесению раздач.
\end{itemize}

Помните, что отмена раздачи является исключительно техническим моментом, нельзя отменять раздачу, если по состоянию стола уже невозможно понять, как именно нужно корректно внести раздачу. По этой причине игроки обязаны посмотреть на предварительный просмотр внесенной раздачи и внести правки заранее. Замешивать тайлы до того, как была внесена раздача, не следует. В исключительных ситуациях, если тайлы уже замешаны, но все игроки точно помнят как именно нужно внести раздачу, судья может согласиться пойти на отмену раздачи (например, если результаты были озвучены и показаны, но внесены не полностью или не так, как было озвучено).

\subsection{Игры с публикацией на стриминговых сервисах}

При использовании электронных ассистентов на крупных турнирах организаторы могут предложить организацию показательных игр с публикацией их на стриминговых сервисах (twitch).

Как правило, на стрим играет только первый стол (находящийся в топе рейтинга).

\subsubsection{Договоренности с организатором трансляции}

Организатор трансляции должен договориться с организаторами турнира включении в регламент пункта о том, что игроки первых столов соглашаются на стрим. Организатор трансляции при этом вправе выбрать, кто будет играть на камеру.

На трансляцию попадает рука только одного игрока за столом. Игрок вправе отказаться играть под камеру, если все игроки стола откажутся --- организатор трансляции может предложить вывести на трансляцию второй стол из рейтинга, либо решить не транслировать стол вообще.

В случае, если предполагается транслировать руки всех игроков, требуется получить согласие каждого игрока.

Трансляция игрового стола должна происходить с задержкой минимум в 5 минут для исключения возможности подглядывания/подслушивания.

\subsubsection{Дополнительные требования к играющему на трансляции}

Игрок, чья рука транслируется на стрим, должен придерживаться следующих правил:
\begin{itemize}
	\item Поскольку камера находится слева от руки игрока, недопустимо загораживать обзор своей левой рукой;
	\item Сброс тайлов желательно также осуществлять правой рукой;
	\item Руку требуется держать по центру камеры, без смещения влево или вправо. Камера должна быть настроена таким образом, чтобы центр обзора камеры совпадал с центром обзора игрока.
	\item При взятии тайла со стены, необходимо показать его на камеру. Допускается поставить тайл справа от руки, либо положить его боком сверху от руки. 
	\begin{itemize}
		\item В случае, если игра происходит на автоматическом столе, класть тайл сверху руки не рекомендуется, чтобы случайно не показать его игрокам за столом (т.к. тайлы на автостоле магнитные и положенный сверху тайл может случайно повернуться из-за этого).
	\end{itemize}
	\item Дискард следует держать в идеальном состоянии --- тайлы должны быть выложены в три ряда по шесть тайлов, без промежутков между тайлами. 
	\item В случае объявления риичи, палочку-ставку следует класть непосредственно перед первым рядом дискарда.
\end{itemize}

\subsection{Форс-мажоры}

\subsubsection{Некорректная рассадка}

Перед началом игры игроки \textbf{обязаны убедиться}, что место каждого игрока соответствует тому, что отображается в ассистенте.

В случае, если была обнаружена некорректная рассадка игроков по ветрам, необходимо немедленно позвать судью и объяснить ситуацию. В случае, если сыграно не более трех раздач, судья может допустить последовательную отмену раздач с дальнейшим внесением их заново по принципу сохранения выигравших и проигравших. Для этого требуется сохранить куда-либо результаты всех сыгранных раздач, после чего последовательно отменить все раздачи и внести их заново. Необходимо быть внимательным при внесении, поскольку лог раздачи содержит записи с некорректным расположением игроков. Если было сыграно более трех раздач, игроки продолжают игру в текущей рассадке, однако судье следует исправить положение игроков относительно друг друга в панели администратора.

Игрокам, севшим неправильно относительно друг друга, выносится вежливое указание быть более внимательными. В случае повторения ситуации с участием одних и тех же лиц, этим лицам может быть также вынесен произвольный штраф на усмотрение судьи.

\subsubsection{Выход из строя систем автоматизированного учета}

В случае, если продолжать турнир невозможно из-за технических неполадок, организаторам турнира следует \textbf{незамедлительно} связаться с администраторами системы учета. 

В худшем случае организаторам необходимо предоставить каждому столу ручку и лист бумаги для доигрывания игры в ручном режиме.

В случае, если сервис неработоспособен в течение длительного времени, организаторам необходимо предоставить игрокам счетные палочки и бланки для записи результатов игр. Формирование рейтинговой таблицы по итогам игр и дальнейшее проведение турнира в ручном режиме ложится на организаторов турнира.

\subsubsection{Выход из строя интернет-соединения WiFi}

Если доступ к интернету через WiFi по каким-то причинам невозможен, внесение раздач оставляется на тех игроков, которые имеют доступ в интернет через мобильную сеть. Игрок не вправе отказаться от внесения раздачи, если он единственный за столом, кто имеет доступ в интернет.

\subsubsection{Отсутствие мобильных устройств у всех игроков за столом}

В том исчезающе редком случае, если ни у одного игрока за столом нет мобильного устройства для внесения раздач, игроки вправе попросить у организаторов или у судейского состава предоставить им такое устройство. В случае, если доступного устройства нет в наличии, допускается попросить мобильное устройство у любого из столов, где таких устройств больше одного. Поиск мобильного устройства ложится на организаторский и судейский состав турнира.

